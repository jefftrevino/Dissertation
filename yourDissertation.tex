%%!TEX TS-program = xelatex
% !BIB TS-program = biber

%
%
% UCSD Doctoral Dissertation Template
% -----------------------------------
% http://ucsd-thesis.googlecode.com
%
%
% ----------------------------------------------------------------------
% WARNING: 
%
%   This template has not endorced by OGS or any other official entity.
%   The official formatting guide can be obtained from OGS.
%   It can be found on the web here:
%   http://ogs.ucsd.edu/AcademicAffairs/Documents/Dissertations_Theses_Formatting_Manual.pdf
%
%   No guaranty is made that this LaTeX class conforms to the official UCSD guidelines.
%   Make sure tha�t you check the final document against the Formatting Manual.
%  
%   That being said, this class has been routinely used for successful 
%   publication of doctoral theses.  
%
%   The ucsd.cls class files are only valid for doctoral dissertations.
%
%
% ----------------------------------------------------------------------
% GETTING STARTED:
%
%   Lots of information can be found on the project wiki:
%   http://code.google.com/p/ucsd-thesis/wiki/GettingStarted
%
%
%   To make a pdf from this template use the command:
%     pdflatex template
%
%
%   To get started please read the comments in this template file 
%   and make changes as appropriate.
%
%   If you successfully submit a thesis with this package please let us
%   know.
%
%
% ----------------------------------------------------------------------
% KNOWN ISSUES:
%
%   Currently only the 12pt size conforms to the UCSD requirements.
%   The 10pt and 11pt options make the footnote fonts too small.
%
%
% ----------------------------------------------------------------------
% HELP/CONTACT:
%
%   If you need help try the ucsd-thesis google group:
%   http://groups.google.com/group/ucsd-thesis
%
%
% ----------------------------------------------------------------------
% BUGS:
%
%   Please report all bugs at:
%   http://code.google.com/p/ucsd-thesis/issues/list
%
%
% ----------------------------------------------------------------------
% More control of the formatting of your thesis can be achieved through
% modifications of the included LaTeX class files:
%
%   * ucsd.cls    -- Class file
%   * uct10.clo   -- Configuration files for font sizes 10pt, 11pt, 12pt
%     uct11.clo                            
%     uct12.clo
%
% ----------------------------------------------------------------------



% Setup the documentclass 
% default options: 12pt, oneside, final
%
% fonts: 10pt, 11pt, 12pt -- are valid for UCSD dissertations.
% sides: oneside, twoside -- note that two-sided theses are not accepted 
%                            by OGS.
% mode: draft, final      -- draft mode switches to single spacing, 
%                            removes hyperlinks, and places a black box
%                            at every overfull hbox (check these before
%                            submission).
% chapterheads            -- Include this if you want your chapters to read:
%                              Chapter 1
%                              Title of Chapter
%
%                            instead of
%                              1 Title of Chapter
\RequirePackage[english=usenglishmax]{hyphsubst}
\documentclass[12pt, chapterheads]{ucsd}
\usepackage{listings}
\usepackage{float}
\usepackage[xetex]{graphicx}
\usepackage{fontspec,xunicode}
\usepackage[justification=centering,font=footnotesize]{caption}
\defaultfontfeatures{Mapping=tex-text,Scale=MatchLowercase}
\setmainfont[Scale=.95]{Palatino}
\setmonofont{Courier New} 
\usepackage{microtype}

\usepackage[backref, backend=biber, citestyle=authoryear-ibid, style=authoryear]{biblatex}
\bibliography{review}
\usepackage{color}   %May be necessary if you want to color links
\usepackage{hyperref}
\hypersetup{linktocpage}
\usepackage{color}   %May be necessary if you want to color links
\usepackage{hyperref}
\hypersetup{
    colorlinks=true, %set true if you want colored links
    %linktoc=all,     %set to all if you want both sections and subsections linked
    citecolor=black,
    linkcolor=black,  %choose some color if you want links to stand out
    urlcolor=black
}

\newcommand{\code}[1]{\scriptsize\ttfamily #1\normalsize\normalfont}

\patchcmd{\bibsetup}{\interlinepenalty=5000}{\interlinepenalty=10000}{}{}

%\usepackage[hyperpageref]{backref} 

%\usepackage[sorting=none, backend=biber]{biblatex} % load the package
%\addbibresource{review.bib} % add a bib-reference file

% Include all packages you need here.  
% Some standard options are suggested below.
%
% See the project wiki for information on how to use 
% these packages. Other useful packages are also listed there.
%
%   http://code.google.com/p/ucsd-thesis/wiki/GettingStarted



%% AMS PACKAGES - Chances are you will want some or all 
%    of these if writing a dissertation that includes equations.
%  \usepackage{amsmath, amscd, amssymb, amsthm}

%% GRAPHICX - This is the standard package for 
%    including graphics for latex/pdflatex.
%  \usepackage{graphicx}

%% SUBFIG - Use this to place multiple images in a
%    single figure.  Subfig will handle placement and
%    proper captioning (e.g. Figure 1.2(a))
% \usepackage{subfig}

%% TIMES FONT - replacements for Computer Modern
%%   This package will replace the default font with a
%%   Times-Roman font with math support.
% \usepackage[T1]{fontenc}
% \usepackage{mathptmx}

%% INDEX
%   Uncomment the following two lines to create an index: 
% \usepackage{makeidx}
% \makeindex
%   You will need to uncomment the \printindex line near the
%   bibliography to display the index.  Use the command
% \index{keyword} 
%   within the text to create an entry in the index for keyword.
%   To compile a LaTeX document with an index the 'makeindex'
%   command will need to be run.  See the wiki for more details.

%% HYPERLINKS
%   To create a PDF with hyperlinks, you need to include the hyperref package.
%   THIS HAS TO BE THE LAST PACKAGE INCLUDED!
%   Note that the options plainpages=false and pdfpagelabels exist
%   to fix indexing associated with having both (ii) and (2) as pages.
%   Also, all links must be black according to OGS.
%   See: http://www.tex.ac.uk/cgi-bin/texfaq2html?label=hyperdupdest
%   Note: This may not work correctly with all DVI viewers (i.e. Yap breaks).
%   NOTE: hyperref will NOT work in draft mode, as noted above.
% \usepackage[colorlinks=true, pdfstartview=FitV, 
%             linkcolor=black, citecolor=black, 
%             urlcolor=black, plainpages=false,
%             pdfpagelabels]{hyperref}
% \hypersetup{ pdfauthor = {Your Name Here}, 
%              pdftitle = {The Title of The Dissertation}, 
%              pdfkeywords = {Keywords for Searching}, 
%              pdfcreator = {pdfLaTeX with hyperref package}, 
%              pdfproducer = {pdfLaTeX} }


\begin{document}
\sloppy


%% FRONT MATTER
%
%  All of the front matter.
%  This includes the title, degree, dedication, vita, abstract, etc..
%  Modify the file template_frontmatter.tex to change these pages.
\include{template_frontmatter}





%% DISSERTATION

% A common strategy here is to include files for each of the chapters. I.e.,
% Place the chapters is separate files: 
%   chapter1.tex, chapter2.tex
% Then use the commands:
%   \chapter{A Contextualized History of Object-oriented Musical Notation}
\section{What is Object-oriented Programming (OOP)?}
\subsection{Elements of OOP}

Object-oriented programming (OOP) may be understood as an alternative to a previously conventional segregation of \emph{data} --- values expressed with numbers --- and \emph{functions} --- procedures that process data. This older model of \emph{procedural programming} emphasizes the way in which a program accomplishes a task, by sending data through a pipeline of processing functions, much as a recipe describes separately the ingredients and procedures necessary to create a certain dish. Procedural programming approaches a problem by first asking: what must be done, and how might these tasks be analyzed into smaller tasks sufficiently specific for the limitations of the utilized programming language? OOP, on the other hand, emphasizes the agents that take part in the process. It approaches a problem by first asking: what are the actors, agents, and objects involved in this task, and how to they communicate with each other and behave (\cite[5]{Wirfs-Brock:1990ys})?

\subsubsection{An Example: Implementing a Counter with Procedural and Object-oriented Programming}
In object-oriented programming, data values become object \emph{states} and the functions that process them become object \emph{behaviors}. This means that values and procedures commonly used together have been grouped together into an object. The following example illustrates this fundamental difference between procedural and object-oriented programming.

The problem of managing of a \emph{counter} --- a value to be incremented or decremented to track some aspect of a system --- arises often in the course of music programming: for example, to track the number of measures created in a program designed to make a specified number of measures, a program might repeatedly perform a measure-creating operation and increment a counter to indicate that a measure has been created, until the counter reaches the specified number of measures. To implement such a counter with procedural programming, the programmer might initialize a variable named ``counter'' to hold a value of ``0,''

\begin{ttfamily}
\begin{scriptsize}
\textbf{int} counter = 0
\end{scriptsize}
\end{ttfamily}

\noindent perform the action to be counted, and then pass the value through an addition function to increment the counter:

\begin{ttfamily}
\begin{scriptsize}
counter += 1
\end{scriptsize}
\end{ttfamily}

\noindent In this way, the program records the number of measures created in the data value ``counter.'' In this example, data and procedure exist separately in the system: the counter's value exists as a variable, named ``counter,'' and the incrementing procedure exists as a separate function, represented by the symbols ``+=''.
\\
In contrast to this procedural approach, for an object-oriented counter implementation, the programmer creates a counter object, with a value that records the object's state (``counter'') and a behavior that increments this value (``inc''):

\begin{ttfamily}
\begin{scriptsize}
\textbf{object} Counter:

   \textbf{value} counter

   \textbf{behavior} inc() : counter = counter + 1 
\end{scriptsize}
\end{ttfamily}

\noindent To interact with this object, the programmer may query the value of the counter:

\begin{ttfamily}
\begin{scriptsize}
Counter.value
\end{scriptsize}
\end{ttfamily}

\noindent And the programmer may increment the counter:

\begin{ttfamily}
\begin{scriptsize}
Counter.inc()
\end{scriptsize}
\end{ttfamily}
\subsubsection{Data Abstraction and Encapsulation}
Humans often abstract ideas and perceptions to emphasize meaningful information and suppress irrelevant detail; for example, a map necessarily contains less detail than the navigated landscape to which it corresponds. (\cite[3]{Wirfs-Brock:1990ys}). Likewise, the true detail of a natural environment does not vanish when one uses a map to navigate, and a map can be viewed as a usefully simplified model that affords purposeful interaction with a relatively more complex environment. When interaction with this simplification governs interaction with the environment itself, the map has become an \emph{interface} to the landscape.

The creation of multiple objects with separate internal memories and behaviors in OOP creates an environment in which one object might duplicate the behavior of another object or lack a behavior that characterizes another object; for the object concept to remain meaningful in an environment of multiple, differing objects, one object's memory and behaviors must be accessible only through interaction with that object (\cite[481]{Cardelli:1985oq}). The limitation that an object's internal memory and behaviors may be accessed only via an interface to that object is called \emph{encapsulation} (\cite[18]{Van-Roy:2004uq}), and the broader methodology of segregating the construction of an object from its use is known as \emph{data abstraction} (\cite{Abelson:1983nx}). Because an object's internal construction may only be accessed via the object's interface, the internal construction of an object's behaviors --- the object's \emph{implementation} --- can change drastically, and, if each implementation maintains the same object interface, many implementations will behave identically; this trait is known in the literature as \emph{polymorphism} (\cite[18]{Van-Roy:2004uq}). (This object-oriented definition of polymorphism should not be confused with the literature's use of the term to classify the flexibility of a language's data type handling, as in \cite[472-480]{Cardelli:1985oq}.) For example, the counter object above could be described as above, or in this alternate implementation:

\begin{ttfamily}
\begin{scriptsize}
\textbf{object} Counter:

   \textbf{value} counter

   \textbf{behavior} inc() : counter += 1
\end{scriptsize}
\end{ttfamily}

\noindent The only difference between these two \code{Counter} implementations is the specific formulation of the \code{inc()} method: in the first implementation, the addition function, in conjunction with an assignment operator, increments the counter's value, while in the second implementation, a single operator, \code{+=}, both increments and assigns the value. This difference remains invisible to the user: regardless of the specific implementation of the counter's methods, the interface to the object remains the same, and the programmer may increment the counter by invoking the increment method (\code{Counter.inc()}).

\subsubsection{Problems of Interfaces: Affordances and Transparency}
An object's interface allows and prevents certain modes of interaction with the object's internal state and behavior. Designer Don Norman proposes the idea of \emph{affordances} to describe the way that an interface's design invites or discourages a certain mode of use (\cite{Norman:2003mz}): boys pull on girls' pigtails, because the shape and height of pigtails affords (invites) pulling. Any technology will afford certain interactions and, consequently, applications of the technology. 

The proliferation of interfaces through data abstraction also presents a trade-off between usability and openness. As Nance and Sargent point out:

\begin{quote}A major consequence of the conjunction of HCI [(Human-Computer Interaction)] with other advances is an ever-increasing user relief from the requirement to have detailed knowledge of the underlying computing technology. The result has greatly expanded the population of productive users of the ubiquitous digital technology. However, a concomitant result is that, unless the user forces revealing actions, the modeling software hides how the function is performed (\cite[164]{nance2002perspectives}).
\end{quote}

\noindent In the trade-off between technological transparency, on the one hand, and straightforwardness of use, on the other, object-oriented programming trades access to object internals for an interface's affordances. 

\subsubsection{Incremental Data Abstraction via Classes and Inheritance}

A set of data abstractions might address most of the necessary tasks of a given application area, but the programmer will likely need to create new abstractions to meet the challenges of new problems or propose novel solutions to established problems; because a programming language might be used to solve novel problems, languages should simplify the process of creating new abstractions (\cite{Liskov:1974rt}). For example, if two objects are similar, it would be useful to create one object with reference to the other, by describing only the difference between the two. OOP enables this with its system of classes and inheritance.

OOP departs fundamentally from other paradigms by abstracting the idea of a data type. (\cite[18]{Van-Roy:2004uq}). OOP creates objects from templates called \emph{classes}. A class describes a kind of object and includes \emph{attributes}, \emph{methods}, and \emph{properties}. When an object is created from a class, it is an \emph{instance} of the class and \emph{inherits} all of the attributes, methods, and properties of the class; \emph{inheritance} defines new abstractions as incremental extensions of existing abstractions and allows the user to create a new object by describing only the difference between the old and new objects. To say that some new class (known variously as a \emph{derived class}, \emph{child class}, or \emph{subclass}) \emph{inherits} from an existing class is to say that the new class contains its own hidden, encapsulated version of the methods, attributes, and properties of the class from which it derives (known variously as the \emph{base class}, \emph{superclass}, or \emph{parent class}). Class attributes might be initialized uniformly, through the specification of a default value in the class definition itself, or individually, when the specific object is \emph{instantiated} from the class. For example, to instantiate an object from the Note class in the Abjad API for Formalized Score Control (Ba\v{c}a, Oberholtzer, and Ad\'{a}n, 1997-present), the programmer must supply a pitch and duration attribute to the \code{Note()}function, which instantiates an object from the Note class:

\begin{lstlisting}[basicstyle=\scriptsize\ttfamily, breaklines=True, tabsize=4, showtabs=false, showspaces=false]
>>> a_note_object = Note( "c'", (1,4) )\end{lstlisting}



\noindent Named a\_note\_object, this object inherits all of the attributes, methods, and properties from its instantiating class:

\begin{scriptsize}
\begin{ttfamily}
note.descendants

note.duration\_multiplier

note.leaf\_index

note.lilypond\_format

note.lineage

note.multiplied\_duration

note.note\_head

note.override

note.parent

note.parentage

note.preduration

note.duration

note.prolation

note.set

note.sounding\_pitch

note.spanners

note.start\_offset

note.stop\_offset

note.storage\_format

note.timespan

note.written\_duration

note.written\_pitch

note.written\_pitch\_indication\_is\_at\_sounding\_pitch

note.written\_pitch\_indication\_is\_nonsemantic
\end{ttfamily}
\end{scriptsize}

\noindent Every note object instantiated from the Note class inherits the same set of attributes from its instantiating class. The arguments given to the \code{Note()} function have supplied values for several of these attributes; for example, this specific object has a written pitch equal to middle C and a duration of one quarter note; this can be seen by querying the object's \code{written\_pitch} and \code{duration} attributes:

\begin{lstlisting}[basicstyle=\scriptsize\ttfamily, breaklines=True, tabsize=4, showtabs=false, showspaces=false]
>>> a_note_object.written_pitch
NamedChromaticPitch("c'")
>>> a_note_object.duration
Duration(1, 4)\end{lstlisting}

\subsection{A Nosebleed History of OOP}
\subsubsection{Cybernetics}
During and in the decade following World War II, scientists formulated mathematical models of communication, cognition, homeostasis, and biological systems (\cite{Aspray:1985uq}). Shortly after the end of the war, \emph{cybernetics} --- a term coined by Norbert Wiener in 1948 --- emerged as an academic field organized around the study of command and control dynamics, the design and analysis of systems, and analogies between organisms and machines, including computers (\cite{Dunbar-Hester2009}). While conducting anti-aircraft weaponry research at MIT's Radiation Laboratory in 1943, Norbert Wiener, Julian Bigelow, and Arturo Rosenblueth bifurcated the analysis of human-machine systems into two paradigms: a ``behavioristic'' model that emphasizes the relationship between a system's inputs and outputs and a ``functional'' understanding that emphasizes an understanding of the internal structure and function of objects (\cite{Priestley:2011ve}). While the authors were concerned at the time with illustrating how this ``behavioristic'' understanding of systems allows the uniform analysis of human-machine systems, thus enabling the unified discussion of systems with both human and machine actors and laying the groundwork for cybernetics, this dichotomy between an input-output systems model and an object-based systems model presages the computational models that underlie procedural and object-oriented programming languages respectively. The presence of this dichotomy in the founding work of cybernetics does not demonstrate a clear line of influence between object-oriented programming and cybernetics; rather, it shows that practitioners have acknowledged since the beginning of computation research that a view of programming abstractions as either participants in a processing chain or as communicating objects with encapsulated construction and behavior can lead to divergent views of systems and problems.

\subsubsection{Simulation}	
	The history of OOP languages begins with the simulation programming languages of the 1960s, of which Simula67 (1967), based heavily on Algol60 (1960), was the first (\cite[489]{Van-Roy:2004uq}). Simulation languages seek to model the behavior of complex systems in order to enhance system performance (\emph{system analysis}), evaluate the performance of systems (\emph{acquisition and system acceptance}), and create artificial environments for research and entertainment (\cite[162]{nance2002perspectives}). Simulation predates computers and began as early as 1777, when Buffon estimated the value of $\pi$ by dropping a needle onto strips of wood --- not far afield from this initial experiment, the first computer simulations were ``Monte Carlo'' models, a technique for modeling complex systems that uses deterministic inputs to constrain and measure a random distribution: i.e., to estimate the value of $\pi$, circumscribe a circle inside a square, place points randomly and uniformly within the square, and then measure the ratio of the number of points inside the circle to the total number of points to derive the value of $\pi$ (\cite[162]{nance2002perspectives}). From the perspective of user interaction, a simulation language allows the user to describe the elements of a system, their attributes, their permissible logical relationships, and the time-dependent processes that govern the behavior of the system (\cite{Kiviat:1993fj}). Simulation languages contribute several key concepts to OOP, as well as to computer science more broadly: Simula (1965) proposed that a section of code represent a ``quasi-independent'' process; SIMSCRIPT II (1968) introduced an \emph{entity/attribute/set} concept, by which entities could be both members of sets and unique objects with their own attributes; and Simula67 extended this model by introducing the key concepts of data types, inheritance, encapsulation, and message passing between entities (\cite[167]{nance2002perspectives}). 

\subsubsection{Structured Programming}
In the 1960s, \emph{structured programming}, in which sequentially specified and grouped operations describe the order in which the program processes data values, became an industry best practice, and professionals warned against programming habits that decoupled the sequence of execution from the order in which procedures have been specified; as Edsger W. Dijkstra recommends in his famous letter to the editor, ``Goto Statement Considered Harmful'':

\begin{quote}
...[W]e should do (as wise programmers aware of our limitations) our utmost to shorten the conceptual gap between the static program and the dynamic process, to make the correspondence between the program (spread out in text space) and the process (spread out in time) as trivial as possible (\cite{Dijkstra:1968ul}).
\end{quote}

Dijkstra continues to decry the use of the ``goto'' statement, a programming device that allows programmers to leap to specified line of code in the written program, thus specifying a sequence of code execution that differs substantially from the written, visual order of commands and makes the analysis and evaluation of programs more difficult. Beyond the specific goal of revising contemporary programming habits, this call for the elimination of a ``gap'' between the static nature of code and the dynamic nature of the data processing it enables sets the conceptual stage for an object-oriented approach that emphasizes the cooperation of variously static or dynamic objects that interact with one another, while enabling programmers to conceptualize a task as a sequence of processing functions that act upon data values.

\subsubsection{Other Influences}
Other research areas and trends contributed to the formation of object-oriented paradigms. Knowledge representation languages (such as KRL, KEE, FRL, and UNITS) for Artificial Intelligence engaged discrete state simulations with a knowledge theory based on Minsky's concept of ``frames,'' while ACTORS and FLAVORS (both 1981) developed message passing and multiple inheritance respectively (\cite{stefik1985object}). Several object systems were added to the LISP language widely used in AI between the late 70s and late 80s (\cite{Bobrow:1986qf}), and the self-defining organization of LISP inspired the definition of object-generating classes as objects themselves in Smalltalk during the 1970s (\cite[575]{Kay:1996vn}). As more efficient time-sharing mainframes defined the metaphors of personal computing in the 1960s, computers modeled users as interacting agents with states and behaviors, and \emph{recursive design} allowed operating systems to model themselves, creating a number of \emph{virtual machines} that each encapsulated the computation abilities of the mainframe computer itself (\cite{Creasy:1981ys}).  Metaphors of time sharing and the ``master''/``instance'' data model of Ivan Sutherland's pioneering drawing program and interface SKETCHPAD (\cite{Sutherland:1964zr}) both influenced the development of Smalltalk in the 1970s (\cite[575]{Kay:1996vn}) --- although no more than speculation, Smalltalk creator Alan Kay was a professional jazz guitarist before entering college, and musical abstractions such as scales and chords may also have influenced the development of OOP (\cite[579]{Kay:1996vn}).

\subsubsection{Smalltalk}
Smalltalk (1980) is the first widely used object-oriented programming language (\cite{Sammet:1991pd}). Created through research directed by Alan Kay at Xerox during the 1970s, many syntactically divergent versions of the language throughout the decade adhered to the same core principles of recursive design:

\begin{enumerate}
\item Everything is an \emph{object}.
\item Objects communicate by sending and receiving \emph{messages}.
\item Objects have their own memory.
\item Every object is an instance of a \emph{class}, which is itself an object.
\item The class holds the shared \emph{behavior} for its instances.
\item Classes are organized into an \emph{inheritance hierarchy}.
\end{enumerate}

\noindent Evoking Dijkstra's ``conceptual gap,'' the creators of Smalltalk introduced their language to the public as the result of design concerned explicitly with elegant discourse between human and computational models of concepts: 

\begin{quote}We have chosen to concentrate on two principle areas of research: a language of description (programming language) that serves as an interface between the models in the human mind and those in computing hardware, and a language of interaction (user interface) that matches the human communication system to that of the computer (\cite{Ingalls:1981kx}).
\end{quote}

\noindent To align the conceptual framework of intercommunicating objects with the syntax of their new language, they created an ``object message'' syntax to emphasize that Smalltalk's code directs the flow of communication from object to object. In Smalltalk, if \code{bob} is an integer, the programmer sends an addition message to \code{bob} to change \code{bob}'s value: 

\code{bob +4} 

\noindent If the previous value of \code{bob} had been 3, the new value stored in the object would be 7; if \code{bob} were a string instead of an integer, with a value of \code{``Meta''}, the new value in \code{bob} might be \code{``Meta4''} (\cite{Ingalls:1978fk}).

\subsubsection{Hybrid Languages}
While Smalltalk is a pure object-oriented language --- everything is an object, including classes --- several \emph{hybrid} programming languages became popular in the 1990s and 2000s. Languages such as C++, Java, Python, and Perl enable OOP but also contain built-in data types, such as integers, lists, and floats (floating-point decimal numbers), that cannot be modified by the user (\cite{Schwarz:1993vn}, \cite{Gosling:2000ly}, \cite{Van-Rossum:2003ys}, \cite{Holzner:1999zr}). These data values must be wrapped in an object instance via inclusion in a class or object attribute or method to participate in the language's class hierarchy; however, the use of such values without OOP enables the basic conventions of structured, procedural programming. Not all of these languages were created to be hybrid languages; Python, for example, began as a completely procedural system and gained object orientation during the course of its development.
\subsubsection{Proposed Modern Standards of OOP}
	New applications of a programming language can result from changes of programming style or personal preference (\cite{Sammet:1991pd}). As OOP has gained hegemony in the programming world, authors have proposed and widely circulated ``mental toolkits'': sets of standards and best practices for programming that maximize the advantages of abstraction, encapsulation, and inheritance without introducing problems at later stages in code development or revision. The practice of any of these models as convention promotes a strict understanding of programming style intended to limit the perils and maximize the benefits of OOP. For example, Robert C. Martin has proposed an mnemonic rule-set for class design, SOLID (\cite{Martin:yq}), and Craig Larman has proposed a similar guide to assigning responsibility to objects and classes, GRASP (\cite{Larman:2002gf}). Such guidelines, should they become standard, can substantially influence both the way in which programmers use a given programming language and the way that programmers approach and design solutions to problems. 

\section{Object-oriented Notation for Composers}

\subsection{Composition as Notation}
Western music's basis in notation implies two kinds of information analysis: graphic specification divides an instruction into perceptually fused but independently specified aspects (pitch, rhythm, dynamic), and segregates event information (notes on the page of the score) from sound production information (instructions for how to play an instrument to produce sound) (\cite[83]{Ariza2005}). While many environments for both notation and sound production have arisen within the last twenty years, the present study concerns itself with systems' efficacy on two fronts: the ability to elegantly express both low- and high-level compositional ideas with the aid of a graphic or text-based programming language and the ability to generate a sufficiently detailed common practice musical notation from the programmed ideas, where sufficiency is assessed with reference to the accepted vocabulary of common practice notational symbols and constructs. Contra taxonomies of computer-aided algorithmic composition (CAAC) environments that have attempted to categorize a much broader set of systems (\cite{Ariza2005b}), the present discussion redefines composition narrowly as the act of programming a computer to create a notation in the form of a document --- although recent practice has shown a healthy willingness to question the technological nature and collaborative context of this document.

Conventional aesthetic assumptions inhibit such a concretist definition of notation, and it is easier to adopt the view that composition is equivalent to notation from the standpoint of a specific aesthetic point of view, advanced in the middle of the twentieth century. While the practice of composition is still conventionally described as a process whereby composers mediate intentions or emotion through a representational technology toward a receiving participant (\cite{Davies:1994qf}), mid-century American composers proposed an alternative approach to the same creative technologies, in which the act of composition exists entirely as the creation of a graphic artifact, despite the assumption that others will pursue sonic responses to the created artifact; as the American composer John Cage asked, ``Composing’s one thing, performing’s another, listening’s a third. What can they have to do with one another?'' (\cite[14]{Cage:2011dq}). This question invites a multiplicity of possible relationships between composer, performer, and listener, and invites the formation of an artistic practice that reconsiders the interrelationship of these three musical roles from first principles. Likewise, Cornelius Cardew's piece, \emph{Treatise} (1963---67), invites the performer to invent correspondences between symbol and action, rather than specifying them via assumed performance practice or explanatory notes. It is in this spirit that the present study circumscribes composition as the creation of a graphic provocation to enacted response, rather than a manifestation of the conduit metaphor of human communication (\cite{Reddy:1979oq}) or a transmitter/receiver model of information (\cite{Shannon:1949tg}).

\subsection{Generative Task as an Analytic Framework}
Software production exists as an organizationally designed feedback loop between production values and implementation (\cite{Derniame:1999fk}), and it is possible to understand a system by understanding the purpose for which it was initially designed, the system's \emph{generative task(s)}. In the analysis of systems created for use by artists, this priority yields a dilemma instantly, as analyses that explain a system's affordances with reference to intended purpose must contend with the creative use of technology by artists: a system's intended uses might have little or nothing in common with the way in which the artist finally uses the technology. For this reason, the notion of generative task is best understood as an explanation for a system's affordances, with the caveat that a user can nonetheless work against those affordances to use the system in novel ways. Generative tasks --- informed by the cultural milieu of software development, economic constraints of software production, and the aesthetic proclivities of artists participating in development processes --- constrain software features to enable a limited subset of possible representations and user interactions.

While composers working traditionally may allow intuition to substitute for formally defined principles, a computer demands the composer to think formally about music (\cite{Xenakis:1992rq}). Keeping in mind generative task as an analytical framework, it is broadly useful to bifurcate an automated notation system's development into the modeling of music and composition, on the one hand, and the modeling of musical notation, on the other. All systems model both, to greater or lesser degrees, often engaging in the ambiguous or implicit modeling of music and composition while focusing more ostensibly on a model of notation, or focusing on the abstract modeling of music without a considered link to a model of notation. Due to the intimate link between notation and musical ideas, it is impossible for a system that models notation to avoid at least implicitly modeling musical and compositional ideas, and a computational model of music and composition is an inevitable component of every automated notation system, even when it exists as an unspoken set of technological constraints. Generative task explains a given system's balance between computational models of music/composition and notation by assuming a link between intended use and system development.

Automation, as the computational execution of a previously human-executed task, complicates a system's evaluation via generative task, because the tasks might be assumed to be executed by either human or computer. A compositional practice that positions notation as a central element of the creative process may claim that drawn notation by the human hand can never be eliminated from the process of composition (\cite[131]{Hiller:1965}), just as well as it may claim that a computer must model in as detailed a manner as possible the typographical palette and choices of compositional thought through notation; that is, after considering generative task, human-computer interaction must be further considered to arrive at a concrete distribution of tasks between human and machine. 

\subsection{Computational Models of Music/Composition}

Computational models of music might entail the representation of higher-level musical entities apparent in the acts of listening and analysis but absent in the symbols of notation themselves, as determined to be creatively exigent. Programming researchers and musical artists have modeled many such extrasymbolic musical entities, such as large-scale form and transition (\cite{polansky1991morphological}, \cite{uno1994temporal}, \cite{dobrian1995algorithmic}, \cite{abrams1999higher}, \cite{Yoo1983}), texture (\cite{Horenstein:2004kx}), contrapuntal relationships (\cite{Boenn:2009oq}, \cite{Acevedo2005}, \cite{Anders:2011kl}, \cite{Balser:1990tg}, \cite{Jones:2000hc}, \cite{uno1994temporal}, \cite{Bell:1995ij}, \cite{farbood2001analysis}, \cite{Cope:2002fv}, \cite{Laurson:2005dz}, \cite{Polansky:2011fu}, \cite{Ebcioglu:1980kl}), harmonic tension and resolution (\cite{Melo2003}, \cite{Wiggins1999}, \cite{Foster:1995qa}), melody (\cite{Hornel:1993mi}, \cite{Smith:1992pi}), meter (\cite{Hamanaka:2005ff}), rhythm (\cite{Nauert2007}, \cite{Degazio:1996lh}, \cite{Collins:2003bs}), timbre (\cite{Xenakis:1991fu}, \cite{Creasey:1996ye}, \cite{Osaka2004}), temperament (\cite{Seymour:2007qo}, \cite{Graf:2006il}), and ornamentation (\cite{Ariza:2003zt}, \cite{Chico-Topfer:1998jl}). This work overlaps fruitfully with analysis tasks, and models of listening and cognition can enable novel methods of high-level musical structures and transformations, like dramatic direction, tension, and transition between sections (\cite[108]{Collins2009}); the overlap of artistic application with analysis and simulation of cognitive models also causes a muddle of various motivations and methodologies, resulting in a field of research without clear evaluative criteria (\cite{Pearce2002}).

It is possible to model music computationally without recourse to the canonized abstractions above. Mid-century artists pioneered the conflation of signal processing and music composition, architecting systems that regarded compositionally both the spectral and symbolic characteristics of sound; in accordance with modernism's interest in coherence between multiple structural scales, temporal scales, and dimensions (\cite{Stockhausen:1962fu}), both Hebert Br\"{u}n and Iannis Xenakis produced composition systems in which larger structural features emerged from mathematical constraints that generate sound files sample by sample (\cite{Luque2009}, \cite{Brun:1969il}). These systems demonstrate that the persistence of older theories of music is optional in computational models of music: in the works of James Tenney and Larry Polansky, for example, ``mean event time'' and probabilistically determined pitch selection algorithms replace traditional musical abstractions, such as tempo (\cite{Polansky:2010fc}).

One subset of these extra-symbolic musical entities are those musical entities that overlap with concepts in notation. For example, a ``chord'' might be a vertically ordered collection of pitch classes in a harmonic conceit, or it might refer to the specific arrangement of pitched noteheads, stemmed together into a composite notation symbol that instructs a performer to perform a sound that consists of several component pitches. Due to substantial overlap in vocabulary between musical and notational concepts, it can be difficult to separate a system's model of music/composition from its model of notation.

A system that affords a detailed model of music/composition without linking it to a sufficiently detailed model of musical notation does not afford automated notation --- sufficiency, however, depends heavily on generative task. For example, if a composer requires an automated notation system to render complex rhythmic ideas that depend typographically on nested tuplets, a system that produces a notation only via a combination of MIDI and quantization must reduce rhythms to a non-hierarchical stream of event times, eliminating the temporally divisive approach of tuplet notation. For many rhythmic applications, though, MIDI suffices. 

\subsection{Computational Models of Notation}

Many automated notation systems exist to model musical notation and the act of typographical layout without explicitly affording the computational modeling of music or composition (\cite{Smith:1972mw}, \cite{Nienhuys:2003ve}, \cite{Hoos:1998bd}, \cite{hamel1noteability}); many of these systems strongly imply a model of music, such as Gr\'{e}goire for Gregorian chant, Django for guitar tablature, and GOODFEEL for Braille notation (\cite{2006}). In this light, feature-rich systems oriented toward classical composers, such as Finale, Sibelius, SCORE, Igor, Berlioz, and Nightingale fit into the mold of systems that model notation with genre as a primary determinant of generative task. Such a system might go so far as to enable a text-based object-oriented model of notation that automates some aspect of an otherwise point-and-click interface, as in the case of Sibelius's ManuScript scripting language (\cite{Technology:qc}). 

Many models of musical notation were created for purposes of corpus-based computational musicology. Formats such as DARM, SMDL, HumDrum, and MuseData model notation with the generative task of searching through a large amount of data (\cite{Selfridge-Field:1997ud}). Commercial notation software developers attempted to establish a data interchange standard for optical score recognition (NIFF) (\cite{niff1995niff}); since its release in 2004, MusicXML has become a valid interchange format for over 160 applications and maintains a relatively application-agnostic status, as it was designed with the generative task of acting as an interchange format between variously tasked systems (\cite{Good:2001if}).

Notation representations that underly many of these GUI-based systems often go undescribed as computer representations of notation, in favor of discussions about human-computer interaction. For example, Barker and Cantor developed an early model of music notation that underlies a four-oscilloscope GUI and describe their work entirely in terms of user interaction (\cite{cantor1971computer}); likewise, discussions of modern commercial notation systems are primarily front-end oriented, without much awareness or criticism of the underlying computational models of notation.

\subsection{Object-oriented Systems}
\subsubsection{The Crucial Development of Hierarchical Models}
Many early models of musical notation were not hierarchical, and Lejaran Hiller, in reflecting on decades of automated notation work, has identified the lack of hierarchical organization as a limitation of early work --- although Nick Collins points out that even Hiller's program PHRASE addresses the hierarchical organization of a score up to the level of a phrase, without moving beyond this mid-level of musical structure to concerns of large-scale form (\cite[108]{Collins2009}). There were several object-oriented music environments by 1990 (\cite[139]{Polansky:1990fk}), most created in or inspired by the newly popular Smalltalk-80 programming language; while they facilitated the hierarchical modeling of musical abstractions, they omitted or radically simplified the hierarchical nature of common notation. For example, Glen Krasner (Xerox Systems Science Laboratory) created Machine Tongues VIII, a music system that created an object-oriented model of the score/orchestra distinction inherited from Max Mathews' Music N languages, with a simple linear model of ``partOn'' and ``partOff'' command sequences (\cite{Krasner:1991uq}), omitting hierarchical organization entirely when the system produces notational output; although subsequent Machine Tongues systems introduced some hierarchical organization via ``note'' objects that inhabited ``event lists,'' systems did not attempt to model the hierarchical detail of all a traditional score's elements. Like Hiller's PHRASE program, Andreas Mahling's CompAss system organized events hierarchically up to the mid-level ``phrase'' level of musical structure (\cite{Mahling:1991qf}). These systems are perhaps best conceptualized as Smalltalk-based interfaces to the MIDI standard: as basic extensions of Smalltalk, they enabled the user to arbitrarily extend the system with new objects, creating a detailed and robust model of music, which was ultimately flattened into a list of noteOn and noteOff commands to be notated or played back via MIDI interface. 

While a hierarchical model of notation and of musical events in time can exist in an entirely object-oriented paradigm, it is possible to observe even in these early systems the need for hybrid procedural/object-oriented approaches for the modeling of musical ideas: somewhat counterintuitively, some of the most important objects in these systems are varieties of transformation, to be enacted upon other objects --- the most important nouns are verbs. HMSL --- a system influenced heavily by James Tenney's work on temporal gestalt perception in music (\cite{Tenney:1980kx}), implemented throughout the 1980s in the Forth language, atop a custom object-oriented extension called ODE (Object Development Environment) --- organizes objects hierarchically according to membership in ``morphs,`` objects that represent morphological changes to be applied to raw data, such as parameterized event data (\cite[139]{Polansky:1990fk}). Likewise, Stephen Travis Pope's MODE (Musical Object Development Environment) included ``line segment'' functions to be applied to event lists to transform the parameters of member objects (\cite{Pope:1991ys}). The central role of transformational objects in these first object-oriented systems presages a later preference for hybrid procedural/object-oriented systems, in which built-in primitive data types --- floating point numbers, strings for representing text, integers --- allowed a variously procedural transformation of data or stateful representation of musical objects. (This tendency might be viewed as the computational persistence of signal generators from modular synthesizers, which allow signals to flexibly modulate other signals.) While procedural programming allows a transformational procedure to be executed, object-oriented programming enables a transformation to exist as an parameterized object, with its own set of attributes. 

By 1989, Glendon Diener's Nutation system (written in Objective C for the NeXT computer) had modeled both musical and notational structure hierarchically through the use of directed graphs (\cite{Diener:1991zr}, \cite{Diener:1991ly}, \cite{Diener:1989ve}). While Diener mentions that users should be theoretically able to extend the system's hierarchical modeling to encompass alternative notation approaches and increasingly detailed models of common notation, the system does not include such a model of notation.

\subsection{Graphical Object-oriented Programming Systems}
Although realtime languages were available for music synthesis and control as early as 1981 (\cite{Mathews:1981lo}, \cite{mathews1983rtsked}), it took until the middle of the 1990s for realtime, graphical programming environments to become widely used (\cite{Puckette:1991hs}, \cite{Puckette:1996kc}). While these systems specialize in either signal processing for synthesis applications or symbolic processing for automated notation applications, there are both extensions of Max/MSP and PD that enable musical notation (\cite{didkovsky2008maxscore}, \cite{Kelly:2011rw}) and extensions of OpenMusic and PWGL that enable synthesis and control of synthesized sound; notably, the specification of scores in tandem with control parameters for sound synthesis was a generative task in the creation of Pure Data.

IRCAM developed the Crime, CARLA, and Patchwork environments for composition in the second half of the 1980s, and PatchWork was the first object-oriented automated notation environment to catch the attention of established composers, including Brian Ferneyhough, G\'{e}rard Grisey, Magnus Lindberg, Tristan Murail, and Kaija Saariaho (\cite{Assayag:1999sw}); IRCAM developed PatchWork further into the OpenMusic environment, which gained, over the course of a decade of development, an interface to the control of synthesis parameters (\cite{agon2000high}), an interface to physical modeling (\cite{polfreman2002modalys}), analysis applications (\cite{buteau2009melodic}), an interface to feature data (\cite{bresson2010processing}), and a collection of third-party libraries that extend the basic ``boxes'' included in the environment distributed by IRCAM. 

The naturalistic aesthetic agenda of spectralism played an important role as a generative task for these extensions. Composers' needs demanded the integration of signal processing and symbolic manipulation, and the SDIF standard, a sound file analysis interchange data representation standard developed jointly by CNMAT and IRCAM in the second half of the 1990s (\cite{wright1999audio}) had been incorporated into OpenMusic with SDIF-specific classes and methods by 1999 (\cite{schwarz2000extensions}). 

As OpenMusic developed, Mikael Laurson, the creator of PatchWork, was independently developing PatchWork into PWGL (PatchWork Graphical Language) (\cite{Laurson:2003fh}), a system quite similar to OpenMusic in its graphical approach. PWGL adopts a fundamentally different stance with regard to computationally modeling the details of musical notation. While OpenMusic requires export to a typography program to make choices beyond pitches and rhythms, PWGL provides ENP (Expressive Notation Package) for composers who want to work computationally with common notation symbols (\cite{kuuskankare2009enp}). (It should be emphasized that this limited model of notation has not prevented composers from successfully realizing their ideas using OpenMusic, as documented in IRCAM's two-volume review of projects created using their environment (\cite{agon2006om}, \cite{Agon:2008xd}).) PWGL also developed an interface to synthesis parameters (\cite{laurson2005pwglsynth}) and analysis data (\cite{Kuuskankare:dq}, as well as an interface for graphic notation (\cite{kuuskankareconnecting}) and constraint programming (\cite{Laurson:2006oa}). The Meta-score graphical editor combines procedural programming, common notation, and timeline-based event specification into a single GUI (\cite{kuuskankare2012meta}).

\subsubsection{Live and Interactive Notation}
Some of the most innovative object-oriented musical notation models have been created for applications in which a notation is generated live in realtime with computer assistance, or a pre-composed notation is presented during a performance by means of computer animation. Harris Wulfson's LiveScore system models notation in the Java programming language via NoteStream objects, which each contain a succession of notes, accompanied by text instructions and dynamic markings; in his composition, \emph{LiveScore}, audience participants tune the knobs of a mixer interface to alter the ranges of musical parameters constraining the output of an algorithmic composition engine (\cite{wulfson2007automatic}, \cite{Barrett:2010dq}). Luciano Azzigotti created a similar system in the Processing environment, a simplified dialect of the Java programming language intended to teach artists and designers basic programming skills (\cite{reas2007processing}, \cite{Azziggoti:2012bh}). Throughout the field of music, increased computation power and programming environments tailored to realtime computation have made it easier for composers to creatively refashion notation to satisfy new goals of collaboration and realtime interactivity (\cite{Balachandran:2012cr}). As these new trends are equally likely to engage abstract animation and data representation traditions of information display as they are traditional musical notation, they tend to result in computer models of notation that offer either a simplified set of common practice notational constructs or a novel approach to notation suited to a particular performance application; these applications are a good example of the way in which the set of generative tasks that interest current practitioners may reduce or discard the full range of accepted common practice notational constructs (\cite{Collins:2011nx}).

These new systems cast a distinctly contemporary light on automated notation systems oriented toward a document preparation model of notation production. Whereas computer notation systems could previously agree implicitly to participate in common practice tradition without argument, the proliferation of new approaches to notation in the realm of interactive media marks document preparation systems for common practice notation as definitively conservative technologies. As such, they conceptualize new technology as an assistive technology that aids, enhances, or re-approaches an established notational technology, as opposed to a force for the creation of a radically new paradigm of musical collaboration through graphic media.

\subsubsection{Constraint Solvers}

The most recent trend in the algorithmic development of automated notation programs has been the integration of constraint solvers, which allow the user to describe the result of a process, without describing the means by which the results must be achieved, a paradigm known in computer science as \emph{declarative programming}. A program consists of a descriptive logic, which specifies what to do, and control, which specifies how to do it (\cite{kowalski1979algorithm}); the former is called \emph{declarative programming} and the latter \emph{imperative programming}. Constraint programming is a form of declarative programming, in which the programmer specifies logical constraints that describe the conditions that must be satisfied, without stating exactly how they will be (\cite[749]{Van-Roy:2004uq}); when coupled with an existing imperative language, constraint solvers enable a kind of meta-programming (\cite{lloyd1994practical}). While constraint-solvers depend traditionally on boolean expressions, recent work has devised constraint-specification syntaxes specific to the needs of musical applications, which include arbitrarily chaining the score elements to which constraints apply, as well as specifying constraints that consider relationships between score elements (\cite{anders2008higher}). Constraint solvers have been used to model specific musical structures, such as melodies (\cite{Zhong2005}) and polyphony (a conflation of harmony/counterpoint) (\cite{Courtot:1990gb}), as well as composition more broadly (\cite{Desainte-Katherine:1991mb}). 

An increase in computational power has facilitated the use of constraint solving for complicated musical decisions, and constraint solvers have been integrated into widely used automated notation environments during the last decade. These systems are now fast enough to use in realtime applications (\cite{Anders:2008cq}), as well as in document-oriented notation applications. OpenMusic and PWGL both contain harmonically oriented constraint solvers, and Strasheela is a powerful text-based constraint solver for musical applications (\cite{sandred2010pwmc}); the developers of Abjad are currently integrating a constraint solver that can be arbitrarily applied to any of the components in a graph tree representation of a musical score (\cite{Baca:2013kh}). For the programming composer, the basics of procedural and object-oriented programming might soon be displaced by the careful description of constraints; this development could potentially lower the bar for composer entry into automated notation, because constraint functions as an interface to the automated arrangement of hidden primitive objects and functions, the low-level manipulation of which need not be mastered by the user in order to produce results that meet the specified constraints. Such a system would present a new incarnation of the trade-off between ``user friendliness'' and technological transparency, potentially minimizing the intricacies of procedural programming for composers.

\section{Design Values for Automated Notation Systems, Illustrated with the Abjad API for Formalized Score Control}
\subsection{The Abjad API for Formalized Score Control}
Abjad is a mature, fully-featured system for algorithmic composition comprising, at the time of writing, more than 178,000 lines of code divided into 50 public packages, 305 public classes, 1003 public functions and a documented API totaling more than 800 pages. The system is not built to implement any one idea of what composition is. Abjad is instead architected in such a way as to encourage composers, music theorists and musicologists to model and implement their own, perhaps highly idiomatic, understandings of what musical score is and how music is to be written, analyzed and understood. After a survey of existing automated notation systems, the author has come to regard it as an example of several desirable design values: it allows the user to navigate complex score hierarchies with a readable syntax and access to both high-level and low-level symbolic manipulations, it contains a sufficiently detailed object model of common practice music notation, in which the user may automate the placement of any of the modeled notational symbols, its second-order relationship to generated notations affords tweakability, and its basis in the Python programming language affords extensibility. This section introduces the Abjad API and elaborates on these design criteria.

\subsubsection{Abjad Wraps Lilypond}
Abjad is a Python API that creates formatted Lilypond syntax for the generation of notation by the Lilypond automated music typsetting engine. LilyPond is an automated music typesetting program, created in the C++ and Scheme programming languages (\cite{Nienhuys:2003ve}). Inspired by the deficiencies of computer typesetting work from the last years of the 1980s, LilyPond represents over a decade of research into a text-based interface for the notational constructs of common practice notation, as well as the typographical details and layout of the score as a document (\cite{Schankler:2013mi}). By providing an interface to a sufficiently detailed low-level model of notation, Abjad provides automated access to all the specifics of the score as a document, including typographical details such as a text indications and articulations, as well as format and layout details such as page size and font details.

As a minimal example, the code below creates an Abjad measure and then both encodes and displays the measure via LilyPond, by using the \code{show} function:

\begin{figure}[H] 
\begin{lstlisting}[basicstyle=\scriptsize\ttfamily, breaklines=True, tabsize=4, showtabs=false, showspaces=false]
>>> measure = Measure((5, 8), "c'8 d'8 e'8 f'8 g'8")
>>> f(measure)
{
    \time 5/8
    c'8
    d'8
    e'8
    f'8
    g'8
}
>>> show(measure)\end{lstlisting}

\includegraphics{images/chapter1-1.pdf}

  \caption{Abjad creates notation by scripting the LilyPond typesetting program. \index{Abjad creates notation by scripting the LilyPond typesetting program.}} 
\end{figure}

\subsubsection{Inception}
The code that was to become Abjad began in 1997 and 1998 as the independent work of composers Trevor Ba\v{c}a and Victor Ad\'{a}n. At that time the composers were working with compositional applications of the electroacoustic techniques of granular synthesis and spectral convolution as well as with matrices and other structures from linear algebra, the use of one- and two-dimensional recursive series to model rhythm, and the use of imaging data from graphic input tablets to model geometric transforms of independent musical parameters. The composers found these and many other ideas from group theory, graph theory and computer science to be imminently compositionally useful. But again and again the barrier to the musical exploration of these ideas was found to be the transcription of these objects into the standard notation of musical score (\cite{Baca:2010fk}). 

Ba\v{c}a and Ad\'{a}n have been joined by composer Josiah Oberholtzer as principle architects of the system, and composers, such as the author, have shaped the design of the open-source system by communicating their needs while designing their own compositional applications. Two examples of this feedback between composition and system design follow. 

\subsubsection{Lid\'{e}rcf\'{e}ny}
Lid\'{e}rcf\'{e}ny is a 15-minute work for flute, violin and piano. The piece is the work of Trevor Ba\v{c}a and was written in 2007---2008.
During the composition of the piece Ba\v{c}a used Abjad to render hundreds of rhythmic structures as a fully notated score. The composer worked iteratively and selected the best results from each round of output for use as input to the next round of work with Abjad. Estimating two and a half handwritten pages of score an hour, this process would have taken four years to complete by hand.

Also important to the construction of the piece was the implementation of the Spanner class. The spanner is a structural component unique to Abjad. Spanners play the role of hierarchy-breaking objects that cross over tree-like parts of the musical score. Ba\v{c}a took inspiration for the Abjad spanner from legal publications that posit the idea of a neomedieval overlapping of legal systems in the emergent transnational institutions of the European Union. The Spanner class is now included in the Abjad public library.

The rhythmic construction of Lid\'{e}rcf\'{e}ny shows how the iterative and transcriptional work that Abjad does well can be leveraged in such a way as to reserve the work of creative elaboration for the composer. And the object-oriented flexibility of Abjad made it possible to combine ideas from computer science and jurisprudence in the writing of a piece of music.

\subsubsection{Aurora}
Aurora is a work for 22-voice string orchestra by Josiah Wolf Oberholtzer. It was commissioned in 2011 by Berlin’s Ensemble Kaleidoskop for a festival commemorating the 10th anniversary of Iannis Xenakis’s death. The composer had two main interests when architecting the piece. First, it should be composed of massed clouds of overlapping material, clouds which could permeate, mask or otherwise be superpositioned relative one another. Second, the atoms comprising those massed clouds would be conceived not principally as streams of pitches and rhythms, but as small series of microgestures built from the conglomeration of classes representing idiomatic string techniques. Each instrumental line in Aurora results from multiplexing the traces from each cloud containing that instrument into a single stream, allowing a performer to participate in different composition processes from moment to moment.

Oberholtzer developed the Abjad timeintervaltrees API to accomplish this large-scale formalization. The interval tree is an ordered collection of absolutely-positioned blocks of time to which arbitrary data can be attached. Interval trees can be scaled, split, shifted and exploded without regard for instrumentation or meter because interval trees model the metascore positioning of musical material. Working with Abjad interval trees allows composers to work with large amounts of material that can be rendered as publication-quality notation later in the compositional process.

\subsection{Design Recommendations}

\subsubsection{A Sufficiently Detailed Model of Notation}
In 1971, Cantor writes, ``A full display editor for music would take years to develop, with unforeseen difficulties along the way. To begin, one should construct and use an editor for the notation of some small repertoire'' (\cite[107]{cantor1971computer}). The perilous recommendation that notation should be modeled gradually, moving on to more advanced constructs later, rather than creating at the outset a representation that enables both simple and complicated notational constructs, has left even 21st-century notation editors with overly simple models of notation. Even the most sophisticated commercial editors, for example, advance a model of music in which the measure acts as a system atom, despite the presence of non-mensural notational constructs in every period of notated musical history. 

\subsubsection{Readable Navigation of a Hierarchically Organized Score}
Because a score is necessarily a hierarchical arrangement of symbols, the user needs fluid access to a robust object system, the various levels of the hierarchy in which the symbols have been organized, and the ability to filter collections of objects based on the comparison of their properties. Anything else is an impoverished interface to the basically symbolic nature of musical notation --- anything else, at least, from a conventional aesthetic viewpoint that prioritizes controlled expression over unpredictability and chance (\cite{Gurevich:2007qe}). In line with the cognitive priorities of Djikstra and Kay, these goals should be accomplished in as conceptually elegant a way possible, with a programming syntax that closely aligns with the patterns of human thought in this area of application. Stated from the perspective of the user, rather than the designer, this means that the user must be able to forget, re-approach, and newly understand the function of code. ``Readability'' is paramount for the execution of large projects and to some extent at every moment of the feedback loop between coding and thought. Different programming languages afford readability differentially: Python's syntax, for example, conflates scope with indentation, enforcing through its syntax a convention of good programming style. The simple presence of structured ``white space,'' such as indentation and blank lines, has been shown to increase code readability more effectively than even comments, the first resort of programmers concerned with documenting their code effectively for others (\cite{Buse:2010uq}). The criss-crossing patch cords of graphical programming languages encourage write-only code, while enabling rapid prototyping, and even the syntax of the most elegant object-oriented languages can decrease the readability of code due to the cumbersome task of navigating a hierarchy of symbols. 

Consider the difficulties of the following common notational task --- that of adding phrasing slurs to groups of notes and chords surrounded on either side by a succession of rests (as in Figure 1.2).

\begin{figure}[h] 
\centering
\includegraphics{images/chapter1-2.pdf}

\caption{Rest-delimited notes and chords.\index{Rest-delimited notes and chords.}} 
\end{figure}

Described as a cognitive process, the task might be described in two simple steps: 1) segregate the symbols on the staff into groups of rest and non-rest symbols; 2) add a slur to each non-rest group of symbols. While this first step takes a matter of seconds for the gestalt grouping abilities of the human perceptual system (\cite{Quinlan:1998ov}), it can be a laborious process for an automated notation system. 

\begin{figure}[h] 
\code{
(let (group) (dolist (voice (collect-enp-objects score :voice))

(dolist (chord (collect-enp-objects voice :chord)) (if (rest-p chord)

(progn (when (cdr group)

(insert-expression (reverse group) expression)) (setq group NIL))

(push chord group)))))
}
\caption{Code to slur groups of rest-delimited notes and chords in PWGL/ENP.\index{Code for slurring note/chord groups in PWGL/ENP.}} 
\end{figure}

In PWGL's ENP, the code in Figure 1.3 adds slurs to rest-delimited, mixed groups of notes and chords. Although the code refers clearly to the elements of common notation --- names like ``chord'' and ``score'' indicate that the user has access to a representation of the score --- its many nested, parenthesized arguments do not elegantly map to the plain-language description of procedures above. This is largely because the code's syntax introduces a collection of metaphors foreign to the two verbs of the simple description --- the succession ``group'' and ``slur'' from the original formulation has become ``collect,'' ``collect,'' ``insert,'' ``reverse,'' ``set,'' and ``push'' --- with the effect of complicating the operation beyond its necessary complexity.

Data abstraction can help reduce syntactic clutter and align the language of code with the language of concept. Abjad's built-in iteration functions leverage Python's ability to iterate through lists of symbols, resulting in the following elegant syntax for the above task:

\begin{figure}[h] 
\centering
\begin{lstlisting}[basicstyle=\scriptsize\ttfamily, breaklines=True, tabsize=4, showtabs=false, showspaces=false]
>>> staff = Staff( r"\times 2/3 { c'4 d' r } r8 e'4 <fs' a' c''>8 ~ q4 \times 4/5 { r16 g' r b' d'' } df'4 c' ~ c'1" )
>>> for group in componenttools.yield_groups_of_mixed_klasses_in_sequence(staff.leaves, (Note, Chord)):
...     spannertools.SlurSpanner(group[:])
... 
SlurSpanner(c'4, d'4)
SlurSpanner(e'4, <fs' a' c''>8, <fs' a' c''>4)
SlurSpanner(g'16)
SlurSpanner(b'16, d''16, df'4, c'4, c'1)
>>> show(staff)\end{lstlisting}

\includegraphics{images/chapter1-3.pdf}

\caption{Slurred groups of rest-delimited notes and chords.\index{Slurred groups of rest-delimited notes and chords.}} 
\end{figure}

Using a ``for'' loop, Python groups the leaves (rests, skips, notes, and chords) of a staff container (the function's first argument) by segregating leaves into groups that consist exclusively of notes and chords and groups that do not (the second argument), and then iterates through each note/chord group, slurring each. The language of the code aligns well with the plain language task description, and there are essentially two operations: ``group'' (or more accurately, ``for group in groups,'') and ``slur.'' 

\subsubsection{Transparency Affords Tweakability and Extensibility}

Low-level control over typographical detail and high-level procedural manipulation are seldom found in the same system: many commercial notation programs offer exquisite low-level interfaces without any high-level procedural abilities, and many of the most widely used systems offer an impoverished set of or interface to the symbols of common practice notation. Integrated low- and high-level control over notational symbols encourages two important benefits of technological transparency in the input and output of the system: \emph{extensibility} and \emph{tweakability}. Extensibility assumes that understanding of the low-level construction of a system might enable extensions to the system, to afford new applications or more flexible alignment between thought and individual programming style, while tweakability --- related especially to the system's output --- allows the programmer to engage in low-level manipulations of materials generated by higher-level specifications.

Tweakability is a persistent issue in automated notation systems, and many notation systems create workflows that invite the user to address either high- or low-level symbolic manipulations. Because Abjad wraps LilyPond and extends an interpreted language that can be used live in a terminal, the system affords a wide variety of uses. Adopting the paradigms of creativity offered by McLean and Wiggins (\cite{McLean2010}), a ``planner'' user might elect to write a program that generates the entire composition, as is the case with Josiah Oberholtzer's string ensemble composition, \emph{Aurora} (\cite{Oberholtzer:2010kx}), while a ``bricoleur'' user might generate LilyPond syntax live in the interpreter, copying and pasting into a LilyPond document to be meticulously tweaked at a note-by-note level. The system affords note-by-note composition, totalized algorithmic composition, and many hybridized approaches between these extremes.  

Extensibility is an important design value, both as it applies to the user's ability to extend a system and the ability of a system to integrate diverse, extant modules of code. The relevance of extensibility to the user's experience depends heavily on the difference between the system interface offered to its programmers and to its users: if users engage the system with the same knowledge model as programmers, extensibility in this sense is highly relevant; for systems with a large knowledge asymmetry between programmer and user, extensibility has been romanticized to the point of assuring that amateur programmers will be able to achieve expert results (\cite{Standish:1975gd}). Many automated notation environments assume that their users are programming composers rather than composing programmers, and a large information asymmetry often exists between programmer and user. Even in modern systems touted as object-oriented, users cannot take advantage of the ability to create new classes in the system, because the system's documentation is oriented exclusively toward the use of existing system classes, rather than their extension or modification. 

A second understanding of extensibility is perhaps more relevant to programming for artistic applications. In an age with a surfeit of extant code, the concept of extensibility can be rehabilitated as an assessment of a system's ability to integrate modules of code written in diverse languages and for diverse applications. Such an evaluation is especially relevant to artistic creativity, a realm of activity fraught with interdisciplinary bricolage; for example, recent work in computer-aided algorithmic composition proves that the practice of borrowing concepts and mathematical equations from scientific fields for novel musical applications remains alive and well (\cite{Magnus2010}; \cite{Washka:zp}; \cite{Zad2005}; \cite{Acevedo2005}; \cite{Gartland-Jones2003}; \cite{Phon-Amnuaisuk1999}; \cite{Wiggins1998a}; \cite{MIRANDA2007}; \cite{Burraston2004}; \cite{Kroger}; \cite{Laine1988}; \cite{Hornel}; \cite{Melo2003}; \cite{Spicer2004}; \cite{Luque2009}; \cite{Peters2010}; \cite{essl2006circle}). 

With a priority of freely integrating ideas and code from disparate realms of inquiry, languages can be meaningfully evaluated as relatively disciplined. For example, the LISP programming language remains popular in the fields of Artificial Intelligence, Linguistics, and Music, while science and design disciplines have embraced modern object-oriented languages. Given the manifold needs of programming artists, successful integration is requisite for a suitably flexible environment, and a system's interoperability and breadth of use play an important role in the artistic limitations of a system; the Python language, for example, has demonstrated success as ``glue'' between various languages and application domains (\cite{Sanner:1999rp}). As a language gains a reputation for flexible interoperability, programmers create utilities for this language that further increase the language's abilities in this realm (\cite{beazley1996swig}).

The openness of a software environment also constrains extensibility, and the interfaces of open-source software development encourage software extensibility. Fees and licenses can prevent users from making valuable contributions. For example, although IRCAM's OpenMusic is open source in one sense --- the code can be freely downloaded and revised --- the language's basis in LispWorks Common Lisp requires that third party developers buy a license to compile OpenMusic for the testing of additional libraries; this has been a notable barrier to the anarkomposer project, an open-source tool for flexible input/out across automatic notation systems (\cite{Echevarria:2013yj}, \cite{:km}). As an entirely open-source project in an online code repository, with (Sub)version (SVN) version control and class docstrings that integrate code testing and documentation via Sphinx/ReST (\cite{:sphinxReST}), Abjad streamlines the process of user contribution. 

\subsubsection{Enabling High-/Low-level and Procedural/Object-oriented Automation}
By allowing as much technological transparency as possible, with the goal of ensuring extensibility and tweakability, a system should allow a composer to specify both higher-level relationships that lead to notated results and lower-level procedures that place each notational symbol, one at a time, as desired. (It is conceivably the case that a notational practice might consist entirely of the formalization of the number and position of musical symbols, as has been suspected of the composition of Erik Satie's \emph{Vexations} (\cite{Orledge1998}). To solve a given problem, a programmer might think of a number of interacting agents, a mill-like succession of inputs and outputs, or some combination of the two, all embroiled in the infinite field of metaphor that informs and underlies human thought (\cite{Lakoff1980}). This is especially true in musical thought, in which entire theories of music can theorize the same basic elements as dynamic processes or stable entities (\cite{Berger:1994bs}). This fluidity makes it essential to enable meta-formalizations that place procedural and object-oriented conceptions of the same material into discourse with one another; for example, a sequence of pitches might be considered in time, as an unfolding pattern, as well as out of time, as a structure with characteristics, and the outputs of these two models might usefully inform one another to create contextually aware processes (\cite{Hedelin2008}); multiple statistical models of the same musical elements might influence the production of the notation (\cite{Pearce2005}). For this reason, the musical programmer should be able to accomplish tasks with a flexible mix of procedural and object-oriented programming, as afforded by hybrid programming languages like LISP, Java, and Python. While several systems allow the user to define new procedures, based on built-in objects and classes, it remains unclear in most systems how the user might instantiate new classes. 

The performance of contemporary music can be a complicated physical task, and notation often describes physical gesture or position through the use of tablature (\cite[143]{Rastall:1983zr}); recent work has begun to integrate the constraints of the human hand with the composition of music in automated contexts (\cite{Truchet:2004ys}). As an example of low-level typographical control enabled by both procedural and object-oriented thought, consider the creation of an alternative woodwind fingering diagram, as required for the description of multiphonic sounds in contemporary composition (\cite{Backus:1978fv}).   

\begin{figure}[h] 
\centering
\includegraphics{images/chapter1-4.pdf}

\caption{A multiphonic notation, including a woodwind diagram.\index{A multiphonic notation, including a woodwind diagram.}} 
\end{figure}

As important as the visual depiction of physical contact with instruments has become for contemporary notation practice (\cite{Alberman:2005dz}, \cite{Cassidy:2004fu}, \cite{Kanno:2007kl}), no automated notation system has developed an interface for creating woodwind diagrams. For this reason, the user must extend the system by writing new code; however, in most systems, access to document preparation and low-level typographical operations remain too hidden to allow the user to do this. Because Abjad wraps the LilyPond typesetting package, the author was able to create a new Python interface to Mike Solomon's LilyPond woodwind diagrams (2010). As demonstrated in the code appendix, the above diagram can be implemented variously as a procedure that acts on lists of keys to depress, written with only built-in string manipulation functions from Python's standard library (\ref{sec:wwFunction}), as a procedure written more economically with scheme syntax functions from Abjad's schemetools library, (\ref{sec:wwSchemeFunction}), and lastly with object-oriented programming as a documented WoodwindDiagram class (\ref{sec:wwClass}).

In addition to the above extensibility, the system preserves tweakability: the LilyPond format of the above diagram is easily accessed for copying, pasting, and tweaking, using the \code{f()} (format) function:

\begin{lstlisting}[basicstyle=\scriptsize\ttfamily, breaklines=True, tabsize=4, showtabs=false, showspaces=false]
>>> f(fingering)
\woodwind-diagram #'clarinet #'((cc . (one two three five)) (lh . (R thumb)) (rh . (e)))\end{lstlisting}


\noindent This notational interface is relatively automated, in that it creates a diagram representing all of the instruments keys, without demanding that the user specify the positions of each constituent filled or unfilled shape; however, the user retains control of low-level visual details, with the use of graphic overrides, and can alter the symbolic or graphical representation of the instrument's keys (the \code{graphical} markup command), the size of the diagram (the \code{size} markup command), and the thickness of the lines used to render the diagram (the \code{thickness} markup command):
\begin{lstlisting}[basicstyle=\scriptsize\ttfamily, breaklines=True, tabsize=4, showtabs=false, showspaces=false]
>>> not_graphical = markuptools.MarkupCommand('override', schemetools.SchemePair('graphical', False))
>>> chord = Chord("e' as' gqf''", (1,1))
>>> fingering = instrumenttools.WoodwindFingering('clarinet', center_column=['one', 'two', 'three', 'four'], left_hand=['R','cis'], right_hand=['fis'])
>>> diagram = fingering()
>>> graphical = markuptools.MarkupCommand('override', schemetools.SchemePair('graphical', False))
>>> size = markuptools.MarkupCommand('override', schemetools.SchemePair('size', .5))
>>> thickness = markuptools.MarkupCommand('override', schemetools.SchemePair('thickness', .4))
>>> markup = markuptools.Markup([graphical, size, thickness, diagram], direction=Down)
>>> markup.attach(chord)
Markup((MarkupCommand('override', SchemePair(('graphical', False))), MarkupCommand('override', SchemePair(('size', 0.5))), MarkupCommand('override', SchemePair(('thickness', 0.4))), MarkupCommand('woodwind-diagram', Scheme('clarinet'), Scheme([SchemePair(('cc', ('one', 'two', 'three', 'four'))), SchemePair(('lh', ('R', 'cis'))), SchemePair(('rh', ('fis',)))]))), direction=Down)(<e' as' gqf''>1)
>>> staff = Staff([chord])
>>> contexttools.InstrumentMark('Bb Clarinet', 'clar.')(staff)
InstrumentMark(instrument_name='Bb Clarinet', short_instrument_name='clar.')(Staff{1})
>>> score = Score([staff])
>>> lilypond_file = lilypondfiletools.make_basic_lilypond_file(score)\end{lstlisting}


\begin{figure}[h!] 
\centering
\includegraphics{images/chapter1-5.pdf}

\caption{Graphic overrides change the appearance of a woodwind diagram.\index{Graphic overrides change the appearance of a woodwind diagram.}} 
\end{figure}

\subsubsection{Document Preparation}
When evaluated by a conservative schema --- that an OOP system for notation should provide an object-oriented interface to as thorough a model of common practice notation as possible and should enable the composer to control algorithmically the layout and formatting of the score as a printable document, before proceeding to newer models of composer-technology interaction --- most systems perform poorly. Musical notation is graphic, and manual control over the visual aspect of a notation is necessary from the outset (\cite{Dannenberg:1993bh}). A system should be able to cleanly bridge the gap between composing and document preparation, with algorithmic control over the parameters of the document; however, much of the automated notation work from the last twenty years postpones the most important typographical and formatting choices until after translation into a format appropriate for a musical typesetting program, requiring an export to MIDI, MusicXML, or Lilypond before choices about dynamics, articulations, document formatting, and document layout can be specified. This is not a problem if composition is fundamentally the determination of pitches and rhythms; if anything besides these two musical parameters could potentially occupy a status other than decoration, then an alternative approach might be necessary.

\subsubsection{Conclusion --- Plurality and Fluidity of Generative Task Complicate the Evaluation of System Design}
Design recommendations for user interaction and feedback remain elusive, despite the above survey, because systems are designed with various applications in mind. For this reason, the range of available user interfaces varies from score modeling systems that produce no notation and rely on auditory feedback for the results of text input, such as Andrew Sorensen's \emph{impromptu} language for live coding (\cite{Sorensen:2013ij}), to the conventional point-and-click paper simulations of commercial notation software. Should an automated notation system include an interface to a synthesis engine? Yes, according to BACH and PWGL, and no, according to many other systems. This multitude of implied generative tasks complicates comparative system evaluation of interaction and feedback. In addition to potential applications, aesthetic predilection plays a role, too, and one that extends beyond music and into the process of composition immediately: for example, it is arguably more desirable to compose music with the predictive feedback of imagination alone, despite the ready availability of computer applications that create ``mock-ups'' of a composer's work (\cite{Morris:cr}).

A system's generative task can be both plural and fluid. It might be plural if developers hold different concepts of the system's intended use but can agree sufficiently on a certain set of primitives that must be included. In commercial systems, profit becomes a generative task, and the changing demands of a user-base create a constantly shifting agenda for development. It is also the case that systems created for the work of single authors can be suddenly redirected at larger groups of users, causing a radical change in the direction of development but leaving the indelible fingerprint of the system's earlier goals.

Most broadly, these automated notation systems rest upon a common generative task: drawing. But in the same way that order eliminates noise and neutralizes poltical unrest (\cite{Attali:1985ss}), automated notation systems insist that musical notation resembles the symbolic arrangements of language more than drawing or painting, despite a fifty-year-old tradition of deliberately ambiguous relationship between music and abstract graphic art (\cite{Evarts:1968ff}, \cite{Cardew:1961lh}). The systems described here encroach upon the expressive potentials of drawing's analog creativity: a digital, inherently parametric control has usurped the analog control of the human hand's representative capacities. In this sense, an automated notation system cannot help but be impoverished, relative to the graphic potential of physically enacted representation --- but automation excuses itself by hoping to derive benefits orthogonal to those of drawing. Automation eats and metabolizes drawing, to fuel a marathon of symbolic processing. As Glendon Diener writes, 

\begin{quote} Striking the delicate balance between \emph{structural organization} on the one hand and \emph{graphical generality} on the other is a major issue in the design of common music notation systems. The problem, described by Donald Byrd as the ``fundamental tradeoff'' between \emph{semantics} and \emph{graphics} (Byrd, 1986), is readily understood by imagining musical versions of the 'draw' and 'paint' programs available on many small computers. A musical draw program could facilitate high-level editing and performance operations by means of the data structure analogs of notes, staves, parts, and the like, but as a consequence would limit its visual universe to some finite collection of pre-defind symbols. By contrast, a musical paint program by imposing no further organization on its data than that of a two-dimensional array of pixels, would gain graphical generality at the expense of its ability to perform musically meaningful operations on that data (\cite{Diener:1989ve}).\end{quote}

This dilemma suggests a final qualitative criterion by which one may evaluate an automated notation system: an automated notation system implicitly models the range of graphic variability required by composers and proposes a point of balance in a requisite compromise between semantics and graphics.

%   \chapter{Computational Modeling as Analysis}
\section{The Conflation of Analysis and Composition Reveals and Posits Construction}

Formalized score control conflates the analysis and composition of music: now that the abstract formulation of musical order precedes its instantiation as a notation, it is possible to analyze a composer's code to arrive at new insights about the structure of the music and the cognitive processes at play during composition. The creative process gains increased transparency via two routes: first, because of the text-based nature of code, formalized score control reveals the role of metaphor in the creative process; second, a musical analysis may be tested by implementation, as a valid declarative logic implemented as an imperative program that recreates the score. More radically, a musical analysis might be created first as an imperative program that recreates a score. Two examples of this last approach, in which an analysis proceeds first as an imperative sequence of commands, iteratively revised with the goal of recreating a score, are discussed for the remainder of this chapter, after a brief discussion of the variety of metaphors encountered in composers' programs.

\subsection{Formalization Reveals Metaphor}
A study of Java programmers revealed programs built upon metaphors of explored spatial locations, sentient beings, dancing symbols, buzzing sounds created by absent code, graphic-mathematical transformations, mechanical apparatus, and conversations with intelligent agents (\cite{blackwell2006metaphors}), and correspondence with programming composers reveals a plethora of idiosyncratic formulations that underly automated notation programming. These include quantifying metaphors, in which a composer invents a quantitative system to specify numerically a previously qualitative dimension of music, such as Clarence Barlow's systematic formalization of acoustic consonance and dissonance via ``indigestibility functions'' (\cite{Barlow:2011xz}) or Pablo Cetta's similar work (\cite{Cetta:1011jb}); midwife metaphors, which allow a user community to describe the use of a notation system, whether or not the metaphor appears in the code itself, such as the use of ``pouring'' notes into containers in the Abjad documentation (\cite{Baca:2011xi}); score metaphors, in which the composer improvises an object-based system to track the division of a work into sections, such as Jos\'{e} Lopez-Montez's division of his composition, \emph{Autoparaphrasis}, into summarizing procedures --- ``...granulated sound...explosion...mega-trumpet...granulation in decomposition...hyper-acute... intermediate...disintegration...recapitulation'' (author's translation from the Spanish) --- which describe the sequence of events in the work (\cite{Lopez-Montes:2011mq}); the graphic metaphors of visual programming environments, in which the spatial arrangements of objects organize and communicate data flows; the built-in prescriptive metaphors of programming languages, such as ``flattening a list'' to remove embedded parentheses; and disciplined metaphors, imported into code via an academic discipline, such as the use of ``tree'' and ``leaf'' from graph theory.

\section{Reverse Engineering as Analysis: Two Case Studies in Formalized Score Control as Analysis}
\subsection{\emph{Cantus in Memory of Benjamin Britten} (1977-1980) by Arvo P\"{a}rt}
\subsubsection{The Composition}
\emph{Cantus in Memory of Benjamin Britten} by Estonian composer Arvo P\"{a}rt was composed from 1977 to 1980 and published by Universal Edition in 1980 (\cite{Part:1980fk}). The composition was originally conceived as a series of simple rules governing scale descents and durational relationships between parts, recorded on a napkin during a train ride (\cite{Cope:2010uq}).

\subsubsection{The Approach}
The task of creating a program that would generate the published score exactly was approached as a test of the work's origin myth; that is, the working hypothesis from the outset was that the entire composition would be easily expressible using a few functions that act recursively to create a complex effect. As the following code shows, this turned out to be true: a single recursive function creates almost all of the work's pitches and rhythms. This is unsurprising, as an analysis of the score reveals the composition to be a simple prolation canon, in which register correlates to a doubling of duration and a one-octave decrease of register relative to the next higher string voice; all parts descend the a natural minor scale until the work's coda, which cannot be modeled with such straightforward rules. (Subsequent revisions of this code by Josiah Oberholtzer reimplemented much of the procedural code here as object-generating classes and revised many of the procedures to take advantage of some of Python's indigenous idioms, such as dictionaries; this code is part of the Abjad manual and is freely available online at projectabjad.org).

\subsubsection{The Code}

The following code generates the score for Arvo P\"{a}rt's \emph{Cantus in Memory of Benjamin Britten} (1980, following the Universal Edition, Philharmonia Series \#555).

The code begins with a typographical wrapper function, which enables the embedding of custom fonts within Lilypond documents:

\begin{figure}[h] 
\begin{lstlisting}[basicstyle=\scriptsize\ttfamily, breaklines=True, tabsize=4, showtabs=false, showspaces=false]
>>> def fonted(aString):
...     fontString = "\\override #'(font-name . \"Futura\")"
...     outString = fontString + " {"+ aString +"}"
...     return outString
... \end{lstlisting}

\caption{A font function enables custom typefaces. \index{A font function enables custom typefaces.}} 
\end{figure}


Next, the program models the score, beginning with the staffs and their names:

\begin{figure}[H] 
\begin{lstlisting}[basicstyle=\scriptsize\ttfamily, breaklines=True, tabsize=4, showtabs=false, showspaces=false]
>>> bell = Staff([])
>>> contexttools.InstrumentMark( fonted("Campana in La"), fonted("Camp.")  )(bell)
InstrumentMark(instrument_name='\\override #\'(font-name . "Futura") {Campana in La}', short_instrument_name='\\override #\'(font-name . "Futura") {Camp.}')(Staff{})
>>> 
>>> violin1 = Staff([])
>>> contexttools.InstrumentMark( fonted("Violin I"), fonted("Vl. I") )(violin1)
InstrumentMark(instrument_name='\\override #\'(font-name . "Futura") {Violin I}', short_instrument_name='\\override #\'(font-name . "Futura") {Vl. I}')(Staff{})
>>> 
>>> violin2 = Staff([])
>>> contexttools.InstrumentMark( fonted("Violin II"), fonted("Vl. II") )(violin2)
InstrumentMark(instrument_name='\\override #\'(font-name . "Futura") {Violin II}', short_instrument_name='\\override #\'(font-name . "Futura") {Vl. II}')(Staff{})
>>> 
>>> viola = Staff([])
>>> contexttools.InstrumentMark( fonted("Viola"), fonted("Va.") )(viola)
InstrumentMark(instrument_name='\\override #\'(font-name . "Futura") {Viola}', short_instrument_name='\\override #\'(font-name . "Futura") {Va.}')(Staff{})
>>> contexttools.ClefMark('alto')(viola)
ClefMark('alto')(Staff{})
>>> 
>>> cello = Staff([])
>>> contexttools.InstrumentMark( fonted("Cello"), fonted("Vc.") )(cello)
InstrumentMark(instrument_name='\\override #\'(font-name . "Futura") {Cello}', short_instrument_name='\\override #\'(font-name . "Futura") {Vc.}')(Staff{})
>>> contexttools.ClefMark('bass')(cello)
ClefMark('bass')(Staff{})
>>> 
>>> bass = Staff([])
>>> contexttools.InstrumentMark( fonted("Contrabass"), fonted("Cb.") )(bass)
InstrumentMark(instrument_name='\\override #\'(font-name . "Futura") {Contrabass}', short_instrument_name='\\override #\'(font-name . "Futura") {Cb.}')(Staff{})
>>> contexttools.ClefMark('bass')(bass)
ClefMark('bass')(Staff{})\end{lstlisting}

\caption{Modeling the bell and string staffs and their names. \index{Modeling the bell and string staffs and their names.}} 
\end{figure}

Note that this also includes the appropriate clefs for the staffs, as well as both short and long names for each staff. Next, the program groups the string staffs with a bracket, and adds both a time signature and a tempo; because low-level typographical detail can be adjusted, this step also specifies the space between the instrument name and the left edge of the staff:

\begin{figure}[H] 
\begin{lstlisting}[basicstyle=\scriptsize\ttfamily, breaklines=True, tabsize=4, showtabs=false, showspaces=false]
>>> strings = scoretools.StaffGroup([violin1, violin2, viola, cello, bass])
>>> bell.override.instrument_name.padding = 3
>>> for staff in strings:
...     staff.override.instrument_name.padding = 3
... 
>>> score = Score([])
>>> score.append(bell)
>>> score.append(strings)
>>> contexttools.TimeSignatureMark((6,4))(bell)
TimeSignatureMark((6, 4))(Staff{})
>>> tempo = marktools.LilyPondCommandMark('tempo 4 = 112~120 ')(bell)\end{lstlisting}

\caption{Adding string staffs to a score. \index{Adding string staffs to a score.}} 
\end{figure}

Next, the program models the bell part. This is straightforward using a series of data abstractions that call one another, and the entire part results from the concomitant use of several functions. As the levels of musical structure addressed by the functions grow, the code leaves the realm of notation modeling -- the modeling of a ``bar'' -- and fluidly enters the realm of music modeling -- the modeling of a ``phrase'':

\begin{figure}[H] 
\begin{lstlisting}[basicstyle=\scriptsize\ttfamily, breaklines=True, tabsize=4, showtabs=false, showspaces=false]
>>> def bellBar():
... 	bar = Measure((6,4),"r2. a'2.")
... 	marktools.LilyPondCommandMark("laissezVibrer",'after')(bar[1])
... 	return bar
... 
>>> def restBar():
... 	return Measure((6,4), "r1.")	
... 
>>> def couplet():
... 	return Container([bellBar(), restBar()])	
... 
>>> def bellPhrase():
... 	container = Container([])
... 	container.extend([couplet(), couplet(), couplet(), restBar(), restBar()])
... 	return container
... 
>>> def bellPart():
... 	container = Container()
... 	container.extend(bellPhrase()*11)
... 	container.extend(restBar()*19)
... 	lastBellBar = Measure( (6,4), "a'1.")
... 	marktools.LilyPondCommandMark("laissezVibrer",'after')(lastBellBar[0])
... 	container.append(lastBellBar)
... 	return container
... 
>>> bell.append(bellPart())
>>> 
>>> def sixBarsRest():
... 	restBars = Container([])
... 	restBars.extend(Measure( (6,4), "r1.")*6)
... 	return restBars
... 
>>> for staff in strings:
... 		staff.append(sixBarsRest())
... \end{lstlisting}

\caption{Modeling the bell part. \index{Modeling the bell part.}} 
\end{figure}

Next, the program models most of string material, beginning with the pitch material: by creating a set of functions to generate register-dependent descents down the a minor scale; the resulting function is applied to generate a ``descent reservoir,'' a list of pitches, for each instrument, and then a ``contoured descent'' for each instrument, in which the pitches descend from the top of the scale by one more pitch each time before returning to the top of the reservoir:

\begin{figure}[H] 
\begin{lstlisting}[basicstyle=\scriptsize\ttfamily, breaklines=True, tabsize=4, showtabs=false, showspaces=false]
>>> def descentReservoir(numOctaves,transposition,lastNote):
... 	theKey = contexttools.KeySignatureMark('a', 'minor') 
... 	notes = tonalitytools.make_first_n_notes_in_ascending_diatonic_scale(7*numOctaves+1, key_signature=theKey)
... 	notes.reverse()
... 	for note in notes:
... 		note.written_pitch = note.written_pitch + 12*transposition 
... 	outContainer = []
... 	for note in notes:
... 		if note.written_pitch >= lastNote:
... 			outContainer.append(note)
... 	sequencetools.flatten_sequence(outContainer)
... 	return Container(outContainer)
... 
>>> violin1Res = descentReservoir(3,-1,0)
>>> violin2Res = descentReservoir(2,-1,-3)
>>> vlaRes = descentReservoir(2,-2,-8)
>>> celloRes = descentReservoir(1,-2,-15)
>>> bassRes = descentReservoir(1,-2,-12)
>>> 
>>> def contouredDescent(reservoir):
... 	cd = []
... 	for x in range(len(reservoir)):
... 		cd.append(list(reservoir[:x+1][:]))
... 	cd = sequencetools.flatten_sequence(cd)
... 	return cd
... 
>>> vln1cd = contouredDescent(violin1Res)
>>> vln2cd = contouredDescent(violin2Res)
>>> vlacd = contouredDescent(vlaRes)
>>> cellocd = contouredDescent(celloRes)
>>> basscd = contouredDescent(bassRes)\end{lstlisting}

\caption{Modeling pitch as a series of scalar descents. \index{Modeling pitch as a series of scalar descents.}} 
\end{figure}

The next step in modeling the pitches addresses the relationship between the lower notes of the score's diads and the descending upper notes. Declaratively, this relationship may be described succinctly: ``The lower note of a diad is the a minor arpeggio note equal to or less than the upper note of the diad.'' An imperative version has been created through a cascade of conditional statements (a dictionary structure would be both more efficient and more characteristic of the Python programming language; however, the following has been preserved in the name of readability):

\begin{figure}[H] 
\begin{lstlisting}[basicstyle=\scriptsize\ttfamily, breaklines=True, tabsize=4, showtabs=false, showspaces=false]
>>> def addNearestArpNote(note):
...     pitch = note.written_pitch
...     pitchClass = pitch.named_diatonic_pitch_class
...     if pitchClass == pitchtools.NamedDiatonicPitchClass('a'):
... 	    shadowPitch = note.written_pitch - 5
...     elif pitchClass == pitchtools.NamedDiatonicPitchClass('g'):
... 	    shadowPitch = note.written_pitch - 3
...     elif pitchClass == pitchtools.NamedDiatonicPitchClass('f'):
... 	    shadowPitch = note.written_pitch - 1
...     elif pitchClass == pitchtools.NamedDiatonicPitchClass('e'):
...         shadowPitch = note.written_pitch - 4
...     elif pitchClass == pitchtools.NamedDiatonicPitchClass('d'):
...         shadowPitch = note.written_pitch - 2
...     elif pitchClass == pitchtools.NamedDiatonicPitchClass('c'):
...         shadowPitch = note.written_pitch - 3
...     elif pitchClass == pitchtools.NamedDiatonicPitchClass('b'):
...         shadowPitch = note.written_pitch - 2
...     return Chord( [note.written_pitch,shadowPitch], Duration(1,8) )
... \end{lstlisting}

\caption{Modeling the pitches: a switch system for choosing arpeggio notes. \index{Modeling the pitches: a switch system for choosing arpeggio notes.}} 
\end{figure}

The arpeggio selection function is then applied to the contoured descents, resulting in a sequence of descending diads:
\begin{figure}[H] 
\begin{lstlisting}[basicstyle=\scriptsize\ttfamily, breaklines=True, tabsize=4, showtabs=false, showspaces=false]
>>> def addShadow(cd):
...     shadowed = []
...     notLast = cd[:-1]
...     for note in notLast:
... 	    chord = addNearestArpNote(note)
... 	    shadowed.append(chord)
...     shadowed = sequencetools.flatten_sequence(shadowed)
...     last = Chord( [cd[-1].written_pitch], cd[-1].duration)
...     shadowed.append(last)
...     return shadowed
... 
>>> vln1shadowed = addShadow(vln1cd)
>>> vln2shadowed = addShadow(vln2cd)
>>> vlaChorded = []
>>> for note in vlacd:
... 	chord = Chord([note.written_pitch],note.duration)
... 	vlaChorded.append(chord)
... 
>>> celloShadowed = addShadow(cellocd)
>>> bassShadowed = addShadow(basscd)\end{lstlisting}

\caption{Applying the arpeggio notes to the scalar descents. \index{Applying the arpeggio notes to the scalar descents.}} 
\end{figure}
Next, the program models the rhythmic behavior of the score. Each part alternates between two durations, and each lower string part doubles the durations of the previous part; this is modeled via recursion, a programming technique in which a function calls itself until a terminal condition is reached:

\begin{figure}[H] 
\begin{lstlisting}[basicstyle=\scriptsize\ttfamily, breaklines=True, tabsize=4, showtabs=false, showspaces=false]
>>> def durateDescent(longDuration, shadowedDescent):
...     outList = []
...     for x in range(len(shadowedDescent)):
... 	    if x % 2 == 0:
... 		    chord = Chord(shadowedDescent[x].written_pitches,longDuration)
... 		    outList.append(chord)
... 	    else:
... 		    chord = Chord(shadowedDescent[x].written_pitches,longDuration/2)
... 		    outList.append(chord)
...     return outList
... 
>>> def prolateRecursively(firstDur, multiplier, descentList, outContainer=[],listIndex = 0):
... 	if listIndex == len(descentList):
... 		return outContainer
... 	else:
... 		durated = durateDescent(firstDur * pow(multiplier,listIndex), descentList[listIndex])
... 		firstRest = Rest(firstDur * pow(multiplier, listIndex) *1.5)
... 		duratedContainer = Container([])
... 		duratedContainer.append(firstRest)
... 		for event in durated:
... 			duratedContainer.append(event)
... 		outContainer.append(duratedContainer)
... 		return prolateRecursively(firstDur, multiplier, descentList, outContainer,listIndex = listIndex + 1)
... 
>>> shadowedDescents = [vln1shadowed, vln2shadowed, vlaChorded, celloShadowed, bassShadowed]
>>> duratedDescents = prolateRecursively(Duration(1,2), 2, shadowedDescents)\end{lstlisting}

\caption{Recursively generating most of the string parts' rhythms. \index{Recursively generating most of the string parts' rhythms.}} 
\end{figure}

Finally, all the durations are split cyclically at intervals of six quarter notes, tying across the newly created splits, in order to yield a series of durations that comports with the score's time signature:  

\begin{figure}[H] 
\begin{lstlisting}[basicstyle=\scriptsize\ttfamily, breaklines=True, tabsize=4, showtabs=false, showspaces=false]
>>> for x in range(2):
... 	shards = componenttools.split_components_at_offsets(duratedDescents[x].leaves, [Duration(6,4)], cyclic=True)
... \end{lstlisting}

\caption{Splitting durations cyclically by the duration of one bar. \index{Splitting durations cyclically by the duration of one bar.}} 
\end{figure}

While the above functions model most of the pitches and rhythms of the piece, each part contains an irreducible surplus that must be nonetheless composed out. The next code manually adds the remaining music to each string part:

\begin{figure}[H] 
\begin{lstlisting}[basicstyle=\scriptsize\ttfamily, breaklines=True, tabsize=4, showtabs=false, showspaces=false]
>>> copies = componenttools.copy_components_and_covered_spanners( duratedDescents[0][-20:])
>>> duratedDescents[0].extend(copies)
>>> strings[0].append(duratedDescents[0])
>>> vln1finalSustain = Container([])
>>> for x in range(43):
... 	vln1finalSustain.append(Note(0, Duration(6,4)))
... 
>>> vln1finalSustain.append(Note(0, Duration(2,4)))
>>> tietools.TieSpanner(vln1finalSustain[:])
TieSpanner(c'1., c'1., ... [40] ..., c'1., c'2)
>>> vln1finalSustain.extend([Rest((1,4)), Rest((3,4))])
>>> strings[0].append(vln1finalSustain)
>>> 
>>> copies = componenttools.copy_components_and_covered_spanners( duratedDescents[1][-15:])
>>> copies = list(copies[:])
>>> copies[-1].written_duration = Duration(1,1)
>>> copies.append(Note(-3,Duration(1,2)))
>>> for note in copies:
... 	accent = marktools.articulation = marktools.Articulation('accent')(note)
... 	tenuto = marktools.articulation = marktools.Articulation('tenuto')(note)
... 
>>> duratedDescents[1].extend(copies)
>>> strings[1].append(duratedDescents[1])
>>> vln2finalSustain = Container([])
>>> for x in range(32):
... 	vln2finalSustain.append(Note(-3,Duration(6,4)))
... 
>>> vln2finalSustain.append(Note(-3,Duration(1,2)))
>>> tietools.TieSpanner(vln2finalSustain[:])
TieSpanner(a1., a1., ... [29] ..., a1., a2)
>>> tenuto = marktools.articulation = marktools.Articulation('tenuto')(vln2finalSustain[0])
>>> accent = marktools.articulation = marktools.Articulation('accent')(vln2finalSustain[0])
>>> strings[1].append(vln2finalSustain)
>>> vln2finalSustain.extend([Rest((1,4)), Rest((3,4))])
>>> 
>>> for note in duratedDescents[2][-11:]:
... 	tenuto = marktools.Articulation('tenuto')(note)
... 	accent = marktools.Articulation('accent')(note)
... \end{lstlisting}

\caption{Manual composition of pitches and rhythms after generation. \index{Manual composition of pitches and rhythms after generation.}} 
\end{figure}

Because the durations of the viola part cross bars, it is necessary to split the viola part's durations cyclically at an interval of six quarter notes, to conform to the score's time signature.

\begin{figure}[H] 
\begin{lstlisting}[basicstyle=\scriptsize\ttfamily, breaklines=True, tabsize=4, showtabs=false, showspaces=false]
>>> copies = componenttools.copy_components_and_covered_spanners( duratedDescents[2][-11:])
>>> for note in copies:
... 		if note.written_duration == Duration(4,4):
... 				note.written_duration = Duration(8,4)
... 		else:
... 				note.written_duration = Duration(4,4)
... 
>>> duratedDescents[2].extend(copies)
>>> shards = componenttools.split_components_at_offsets(duratedDescents[2].leaves, [Duration(6,4)], cyclic=True)
>>> bridge = Note(-8,Duration(1,1))
>>> tenuto = marktools.Articulation('tenuto')(bridge)
>>> accent = marktools.Articulation('accent')(bridge)
>>> duratedDescents[2].append(bridge)
>>> strings[2].append(duratedDescents[2])
>>> violaFinalSustain = Container([])
>>> for x in range(21):
... 	violaFinalSustain.append(Note(-8,Duration(6,4)))
... 
>>> violaFinalSustain.append(Note(-8,Duration(1,2)))
>>> tietools.TieSpanner(violaFinalSustain[:])
TieSpanner(e1., e1., ... [18] ..., e1., e2)
>>> tenuto = marktools.Articulation('tenuto')(violaFinalSustain[0])
>>> accent = marktools.Articulation('accent')(violaFinalSustain[0])
>>> strings[2].append(violaFinalSustain)
>>> violaFinalSustain.extend([Rest((1,4)), Rest((3,4))])\end{lstlisting}

\caption{Splitting and finishing the viola part. \index{Splitting and finishing the viola part.}} 
\end{figure}

Likewise, the cello and contrabass parts must be split cyclically by the duration of the meter:

\begin{figure}[H] 
\begin{lstlisting}[basicstyle=\scriptsize\ttfamily, breaklines=True, tabsize=4, showtabs=false, showspaces=false]
>>> duratedDescents[3][-1].written_pitches = [-15,-20]
>>> copies = componenttools.copy_components_and_covered_spanners( duratedDescents[3][-8:] )
>>> for chord in copies[1:]:
... 	chord.written_pitches = [chord.written_pitches[1]]
... 
>>> for chord in copies:
... 	tenuto = marktools.Articulation('tenuto')(chord)
... 	accent = marktools.Articulation('accent')(chord)
... 
>>> duratedDescents[3].extend(copies)
>>> shards = componenttools.split_components_at_offsets(duratedDescents[3].leaves, [Duration(6,4)], cyclic=True)
>>> strings[3].extend(duratedDescents[3])
>>> celloCodetta = Container("a,1. ~ a,2 b,1 ~ b,1. ~ b,1. a,1. ~ a,1. ~ a,1. ~ a,1. ~ a,1. ~ a,2 r4 r2.")
>>> strings[3].append(celloCodetta)
>>> 
>>> duratedDescents[4].pop(-1)
Chord('<c>\\maxima')
>>> duratedDescents[4].pop(-1)
Chord('<c d>\\longa')
>>> duratedDescents[4].pop(-1)
Chord('<c e>\\maxima')
>>> cbFinalChords = [Chord([-8,-20], Duration(32,4)), Chord([-10,-22], Duration(16,4)), Chord([-12,-24], Duration(32,4)), Chord([-13], Duration(16,4)), Chord([-15], Duration(32,4))]
>>> duratedDescents[4].extend(cbFinalChords)
>>> shards = componenttools.split_components_at_offsets(duratedDescents[4].leaves, [Duration(6,4)], cyclic=True)
>>> duratedDescents[4].extend([Rest(Duration(1,4)), Rest(Duration(3,4))])
>>> strings[4].append(duratedDescents[4])\end{lstlisting}

\caption{The cello and contrabass pitches and rhythms composed to completion. \index{The cello and contrabass pitches and rhythms composed to completion.}} 
\end{figure}

The program now models the entire score's pitches and rhythms; the rest of the code models dynamic and technical indications, as well as document formatting. Because it's useful at this point to be able to refer to the score via measure indexes, the program adds measure objects to the previously generated music (note that the previous code generates the entire composition's pitches and rhythms nonmensurally):

\begin{figure}[H] 
\begin{lstlisting}[basicstyle=\scriptsize\ttfamily, breaklines=True, tabsize=4, showtabs=false, showspaces=false]
>>> bellBars = []
>>> shards = componenttools.split_components_at_offsets(score[0].leaves, [Duration(6,4)], cyclic=True)
>>> for shard in shards:
... 	bellBars.append(Measure((6,4),shard))
... 
>>> stringPartBars = []
>>> for staff in strings:
... 	theBars = []
... 	for split in componenttools.split_components_at_offsets(staff.leaves, [Duration(6,4)], cyclic=True):
... 		theBars.append( Measure((6,4),split))
... 	stringPartBars.append(theBars)
... \end{lstlisting}

\caption{Placing previously generated pitches and rhythms into measures. \index{Placing previously generated pitches and rhythms into measures.}} 
\end{figure}

Now that the music can be accessed via measure numbers, the program adds dynamic markings to all the parts, as indicated in the score (without apparent pattern):

\begin{figure}[H] 
\begin{lstlisting}[basicstyle=\scriptsize\ttfamily, breaklines=True, tabsize=4, showtabs=false, showspaces=false]
>>> mark = contexttools.DynamicMark('ppp')(bellBars[0][1])
>>> mark = contexttools.DynamicMark('pp')(bellBars[8][1])
>>> mark = contexttools.DynamicMark('p')(bellBars[18][1])
>>> mark = contexttools.DynamicMark('mp')(bellBars[26][1])
>>> mark = contexttools.DynamicMark('mf')(bellBars[34][1])
>>> mark = contexttools.DynamicMark('f')(bellBars[42][1])
>>> mark = contexttools.DynamicMark('ff')(bellBars[52][1])
>>> mark = contexttools.DynamicMark('fff')(bellBars[60][1])
>>> mark = contexttools.DynamicMark('ff')(bellBars[68][1])
>>> mark = contexttools.DynamicMark('f')(bellBars[76][1])
>>> mark = contexttools.DynamicMark('mf')(bellBars[84][1])
>>> mark = contexttools.DynamicMark('pp')(bellBars[-1][0])
>>> 
>>> mark = contexttools.DynamicMark('ppp')(stringPartBars[0][7][1])
>>> mark = contexttools.DynamicMark('pp')(stringPartBars[0][15][0])
>>> mark = contexttools.DynamicMark('p')(stringPartBars[0][22][3])
>>> mark = contexttools.DynamicMark('mp')(stringPartBars[0][32][0])
>>> mark = contexttools.DynamicMark('mf')(stringPartBars[0][39][3])
>>> mark = contexttools.DynamicMark('f')(stringPartBars[0][47][0])
>>> mark = contexttools.DynamicMark('ff')(stringPartBars[0][56][0])
>>> mark = contexttools.DynamicMark('fff')(stringPartBars[0][62][2])
>>> 
>>> mark = contexttools.DynamicMark('pp')(stringPartBars[1][7][0])
>>> mark = contexttools.DynamicMark('p')(stringPartBars[1][12][0])
>>> mark = contexttools.DynamicMark('p')(stringPartBars[1][13][0])
>>> mark = contexttools.DynamicMark('mp')(stringPartBars[1][25][0])
>>> mark = contexttools.DynamicMark('mf')(stringPartBars[1][34][1])
>>> mark = contexttools.DynamicMark('f')(stringPartBars[1][44][1])
>>> mark = contexttools.DynamicMark('ff')(stringPartBars[1][54][1])
>>> mark = contexttools.DynamicMark('fff')(stringPartBars[1][62][1])
>>> 
>>> mark = contexttools.DynamicMark('p')(stringPartBars[2][8][0])
>>> mark = contexttools.DynamicMark('mp')(stringPartBars[2][19][1])
>>> mark = contexttools.DynamicMark('mf')(stringPartBars[2][30][0])
>>> mark = contexttools.DynamicMark('f')(stringPartBars[2][36][0])
>>> mark = contexttools.DynamicMark('ff')(stringPartBars[2][52][0])
>>> mark = contexttools.DynamicMark('fff')(stringPartBars[2][62][0])
>>> 
>>> mark = contexttools.DynamicMark('p')(stringPartBars[3][10][0])
>>> mark = contexttools.DynamicMark('mp')(stringPartBars[3][21][0])
>>> mark = contexttools.DynamicMark('mf')(stringPartBars[3][31][0])
>>> mark = contexttools.DynamicMark('f')(stringPartBars[3][43][0])
>>> mark = contexttools.DynamicMark('ff')(stringPartBars[3][52][0])
>>> mark = contexttools.DynamicMark('fff')(stringPartBars[3][62][0])\end{lstlisting}


\caption{Adding dynamic markings to parts via measure indexes. \index{Adding dynamic markings to parts via measure indexes.}} 
\end{figure}

Next, the program adds technical and expressive markings to the music, again via measure indexes:

\begin{figure}[H] 
\begin{lstlisting}[basicstyle=\scriptsize\ttfamily, breaklines=True, tabsize=4, showtabs=false, showspaces=false]
>>> markup = markuptools.Markup("\\left-column {div. \\line {con sord.} }", Up)(stringPartBars[0][6][1])
>>> markup = markuptools.Markup('sim.', Up)(stringPartBars[0][8][0])
>>> markup = markuptools.Markup('uniti', Up)(stringPartBars[0][58][3])
>>> markup = markuptools.Markup('div.', Up)(stringPartBars[0][59][0])
>>> markup = markuptools.Markup('uniti', Up)(stringPartBars[0][63][3])
>>> 
>>> markup = markuptools.Markup('div.', Up)(stringPartBars[1][7][0])
>>> markup = markuptools.Markup('uniti', Up)(stringPartBars[1][66][1])
>>> markup = markuptools.Markup('div.', Up)(stringPartBars[1][67][0])
>>> markup = markuptools.Markup('uniti', Up)(stringPartBars[1][74][0])
>>> 
>>> markup = markuptools.Markup('sole', Up)(stringPartBars[2][8][0])
>>> 
>>> markup = markuptools.Markup('div.', Up)(stringPartBars[3][10][0])
>>> markup = markuptools.Markup('uniti', Up)(stringPartBars[3][74][0])
>>> markup = markuptools.Markup('uniti', Up)(stringPartBars[3][84][1])
>>> markup = markuptools.Markup('\\italic {espr.}', Down)(stringPartBars[3][86][0])
>>> markup = markuptools.Markup('\\italic {molto espr.}', Down)(stringPartBars[3][88][1])
>>> 
>>> markup = markuptools.Markup('div.', Up)(stringPartBars[4][14][0])
>>> markup = markuptools.Markup('\\italic {espr.}', Down)(stringPartBars[4][86][0])
>>> shards = componenttools.split_components_at_offsets(stringPartBars[4][88][:], [Duration(1,1), Duration(1,2)]) 
>>> markup = markuptools.Markup('\\italic {molto espr.}', Down)(stringPartBars[4][88][1])
>>> markup = markuptools.Markup('uniti', Up)(stringPartBars[4][99][1])\end{lstlisting}

\caption{Adding technical and expressive markings to parts via measure indexes. \index{Adding technical and expressive markings to parts via measure indexes.}} 
\end{figure}

Scores might contain markings beyond those in Lilypond's model of common notation; many of them can be created by combining existing markings in LilyPond's lexicon. In this score, a successive upbow-downbow marking must be created and applied, by combining the upbow and downbow markings into a single command:

\begin{figure}[H] 
\begin{lstlisting}[basicstyle=\scriptsize\ttfamily, breaklines=True, tabsize=4, showtabs=false, showspaces=false]
>>> def rebow(component):
...     markup = markuptools.Markup('\\concat { \\musicglyph #\"scripts.downbow\" \\hspace #1 \\musicglyph #\"scripts.upbow\" }', Up)(component)
... 
>>> rebow(stringPartBars[0][64][0])
>>> rebow(stringPartBars[1][75][0])
>>> rebow(stringPartBars[2][86][0])\end{lstlisting}

\caption{Defining and using a custom technical marking. \index{Defining and using a custom technical marking.}} 
\end{figure}

Next, a list of measure numbers and a loop add rehearsal markings to the score:
\begin{figure}[H] 
\begin{lstlisting}[basicstyle=\scriptsize\ttfamily, breaklines=True, tabsize=4, showtabs=false, showspaces=false]
>>> rehearsalBars = [7,13,19,25,31,37,43,49,55,61,67,73,79,85,91,97,103]
>>> #use a loop to place the marks. attaching them to the top staff is fine,
... for bar in rehearsalBars:
... 		marktools.LilyPondCommandMark("mark \\default")(bellBars[bar-1][0])
... 
LilyPondCommandMark('mark \\default')(r1.)
LilyPondCommandMark('mark \\default')(r2.)
LilyPondCommandMark('mark \\default')(r2.)
LilyPondCommandMark('mark \\default')(r2.)
LilyPondCommandMark('mark \\default')(r1.)
LilyPondCommandMark('mark \\default')(r2.)
LilyPondCommandMark('mark \\default')(r2.)
LilyPondCommandMark('mark \\default')(r2.)
LilyPondCommandMark('mark \\default')(r1.)
LilyPondCommandMark('mark \\default')(r2.)
LilyPondCommandMark('mark \\default')(r2.)
LilyPondCommandMark('mark \\default')(r2.)
LilyPondCommandMark('mark \\default')(r1.)
LilyPondCommandMark('mark \\default')(r2.)
LilyPondCommandMark('mark \\default')(r1.)
LilyPondCommandMark('mark \\default')(r1.)
LilyPondCommandMark('mark \\default')(r1.)\end{lstlisting}

\caption{Adding rehearsal marks. \index{Adding rehearsal marks.}} 
\end{figure}

The remainder of the program handles page layout and the formatting of the entire score, adjusting the space = time relationship that governs horizontal placement of typographical elements, the spacing and thickness of score elements, the size of the page, the composer, and the title; document layout is also object-oriented, and the code becomes a series of lines setting the attribute values of a LilyPondFile and a Score object:

\begin{figure}[H] 
\begin{lstlisting}[basicstyle=\scriptsize\ttfamily, breaklines=True, tabsize=4, showtabs=false, showspaces=false]
>>> score.set.proportional_notation_duration = schemetools.SchemeMoment(1, 8)
>>> score.override.system_start_bar.thickness = 15
>>> score.override.system_start_square.padding = 3
>>> score.override.system_start_square.thickness = 5
>>> score.override.system_start_bracket.padding = 2.5
>>> score.override.rehearsal_mark.padding = 1.3
>>> score.override.rehearsal_mark.font_name = "Futura"
>>> score.override.script.padding = 0.9
>>> 
>>> lily = lilypondfiletools.make_basic_lilypond_file(score)
>>> lily.global_staff_size = 16
>>> lily.layout_block.ragged_right = False
>>> lily.paper_block.markup_system_spacing__basic_distance = 0
>>> lily.paper_block.markup_system_spacing__basic_distance = 0
>>> lily.paper_block.bottom_margin = 1 * 25.4
>>> lily.paper_block.top_margin = 1 * 25.4
>>> lily.paper_block.left_margin = 1.25 * 25.4
>>> lily.paper_block.right_margin = 1.5 * 25.4
>>> lily.paper_block.paper_width = 11 * 25.4
>>> lily.paper_block.paper_height = 17 * 25.4
>>> lily.header_block.composer = markuptools.Markup( fonted('Arvo Pärt') )
>>> lily.header_block.title = markuptools.Markup( fonted('Cantus in Memory of Benjamin Britten (1980)') )\end{lstlisting}

\caption{Document layout and formatting. \index{Document layout and formatting.}} 
\end{figure}

Finally, the program renders a .pdf file (\ref{sec:cantus}).
\subsection{\emph{Windungen} (1976) by Iannis Xenakis}
\subsubsection{The Composition}
Iannis Xenakis's \emph{Windungen} for twelve cellos was commissioned by the cello section of the Berlin Philharmonic in 1976. The composition of the work has not been carefully documented; however, Xenakis biographer James Harley notes that Xenakis composed the work while exploring basic principles of the branch of mathematics known as group theory (\cite[90]{harley2004xenakis}), the simplest applications of which pertain to basic combinatoric groups of permutations and combinations (\cite{magnus2004combinatorial}). Xenakis's electroacoustic experiments with random walks influenced his acoustic music during the 1970s, and pieces from 1973---1984 made use of Brownian motion and random walks to generate pitch material (\cite{solomos2001unity}). In the author's process of modeling, it became clear that the metaphor of a wave reflecting off a surface --- a central metaphor of Xenakis's stochastic synthesis algorithms (\cite{Luque2009}) --- governed the spatial patterns with which the sounds rotate through the ensemble.
\subsubsection{The Approach}
Unlike the P\"{a}rt, no large-scale formal hypothesis for the composition existed prior to the creation of a rendering program. For this reason, the process of reconstruction became an iterative process of successive approximation, section by section through the score, in which code proposed a model of composition and resulting notation assessed the possibility that the original mode of composition could have resembled the generating model. When the resulting notation deviated from the published score, the code was revised to generate results more like those in the published score --- acknowledging that techniques of constrained randomness may potentially generate substantially different results from the same generative principles. Given the documented historic context of the work, this process was undertaken in the presence of possible applications of simple set-based selection and random walks, which proved fruitful: along the way, it was discovered that events seem to happen in exclusive arrangements, i.e., each of the twelve players enters at a different time in the measure, although they all enter in the same measure, or each of the twelve player plays a different pitch in a certain set of pitches. Despite these general guidelines, the relationship between the published score and its encoded reconstruction was substantially noisier and less determined than that in the case of the P\"{a}rt, due to the presupposed use of stochastic techniques.
\subsubsection{The Code}

The composition begins with diatonic material --- the first three scale degrees of an A major scale --- rotated throughout the ensemble; on the page, this seems to be a graphic rotation through the staffs of the score. First, the program models the rotation of materials through the score's twelve staffs:
\begin{figure}[H] 
\begin{lstlisting}[basicstyle=\scriptsize\ttfamily, breaklines=True, tabsize=4, showtabs=false, showspaces=false]
>>> def make_base_list_of_compressed_rotation_tuples(staffIndexBoundsTuple, rotationBandwidth):
...     lowerBound = staffIndexBoundsTuple[0]
...     upperBound = staffIndexBoundsTuple[1]
...     bitList = range(lowerBound - 1, upperBound + 1)
...     rotations = [ bitList[x:x+rotationBandwidth] for x in range(0, len(bitList) - rotationBandwidth + 1) ]
...     del(rotations[0][0])
...     del(rotations[-1][-1])
...     return rotations
... 
>>> def make_base_list_of_uncompressed_rotation_tuples(staffIndexBoundsTuple, rotationBandwidth):
...     lowerBound = staffIndexBoundsTuple[0]
...     upperBound = staffIndexBoundsTuple[1]
...     bitList = range(lowerBound, upperBound)
...     rotations = [ bitList[x:x+rotationBandwidth] for x in range( len(bitList) - rotationBandwidth + 1) ]
...     return rotations
... 
>>> def make_base_list_of_rotation_tuples(staffIndexBoundsTuple, rotationBandwidth, compressedReflections):
...     if compressedReflections == True:
...         rotations = make_base_list_of_compressed_rotation_tuples(staffIndexBoundsTuple, rotationBandwidth)
...     else:
...         make_base_list_of_uncompressed_rotation_tuples(staffIndexBoundsTuple, rotationBandwidth)
...     return rotations
... 
>>> def mirror_base_list_of_rotation_tuples(rotations):
...     copied = rotations[1:-1]
...     copied.reverse()
...     back = copied
...     rotations.extend( back )
...     return rotations
... 
>>> def make_mirorred_base_list_of_rotation_tuples(staffIndexBoundsTuple, rotationBandwidth, compressedReflections):
...     rotations = make_base_list_of_rotation_tuples(staffIndexBoundsTuple, rotationBandwidth, compressedReflections)
...     rotations = mirror_base_list_of_rotation_tuples(rotations)
...     return rotations
... \end{lstlisting}

\caption{Modeling the rotation of material through the score. \index{Modeling the rotation of material through the score.}} 
\end{figure}

With these functions, it is possible to create a single function that makes a cyclic matrix, specifying which staffs the rotating music should be written on at any given moment in given rotation:

\begin{figure}[H]
\begin{lstlisting}[basicstyle=\scriptsize\ttfamily, breaklines=True, tabsize=4, showtabs=false, showspaces=false]
>>> def make_cyclic_matrix_for_rotation_by_bandwidth(staffIndexBoundsTuple, rotationBandwidth, compressedReflections = True):
...     rotations = make_mirorred_base_list_of_rotation_tuples(staffIndexBoundsTuple, rotationBandwidth, compressedReflections)
...     matrix = sequencetools.CyclicMatrix(rotations)
...     return matrix
... \end{lstlisting}

\caption{A function that returns a matrix of cyclic tuples to specify which staffs the rotating music should be written on. \index{A function that returns a matrix of cyclic tuples to specify which staffs the rotating music should be written on.}} 
\end{figure}


One quirky behavior of the way that Xenakis rotates material throughout the ensemble is that the endpoints of the rotation are often solos, as though the rotating material compresses and expands as it rotates, compressing into the part of a single instrument at the endpoints of rotation:

\begin{figure}[H]
\begin{lstlisting}[basicstyle=\scriptsize\ttfamily, breaklines=True, tabsize=4, showtabs=false, showspaces=false]
>>> def goingUp(staffIndexes):
...     if staffIndexes[0][0] == staffIndexes[1][0]:
...         return False
...     else:
...         return True
... 
>>> def add_bookends_to_staff_indexes(staffIndexes):
...     if goingUp(staffIndexes) == True:           
...         frontBookend = [ staffIndexes[0][1] ]
...         rearBookend = [ staffIndexes[-1][0] ]
...     else:
...         frontBookend = [ staffIndexes[0][0] ]
...         rearBookend = [ staffIndexes[-1][1] ]
...     staffIndexes.insert(0, frontBookend)
...     staffIndexes.append( rearBookend )
...     return staffIndexes
... \end{lstlisting}

\caption{Modeling Xenakis's bookended rotations. \index{Modeling Xenakis's bookended rotations.}} 
\end{figure}

With a model of rotation in place, the music can be modeled as a single staff, to be rotated throughout the ensemble; the single staff can be divided, and portions of it copied to the staffs specified by the two-dimensional staff index matrix described above:

\begin{figure}[H]
\begin{lstlisting}[basicstyle=\scriptsize\ttfamily, breaklines=True, tabsize=4, showtabs=false, showspaces=false]
>>> def pair_tuples_with_splits_using_pitches_from_expr(splits, staffIndexes):
...     splitTuplePairs = [ ]
...     for x in range( len(splits) ):
...         pair = staffIndexes[x], splits[x]
...         splitTuplePairs.append(pair)
...     return splitTuplePairs
... 
>>> def pair_pitches_with_splits(matrix, splits, phaseOffset, soloBookends = False):
...     if soloBookends == True:
...         staffIndexes = [matrix[x + phaseOffset] for x in range( len(splits) - 2 ) ]
...         staffIndexes = add_bookends_to_staff_indexes(staffIndexes)
...     else:
...         staffIndexes = [matrix[x + phaseOffset] for x in range( len(splits) ) ]
...     splitTuplePairs = pair_tuples_with_splits_using_pitches_from_expr(splits, staffIndexes)
...     return splitTuplePairs
... 
>>> def repitch_copy(copied, pitch):
...     for note in iterationtools.iterate_notes_in_expr(copied):
...         note.written_pitch = pitch
...     return copied
... 
>>> def add_split_to_score_by_tuple(split, score, staffTuple):
...     allStaffs = set(range( len(score) ) )
...     soundStaffs = set(staffTuple)
...     silenceStaffs = allStaffs - soundStaffs
...     for x in soundStaffs:
...         copied = componenttools.copy_components_and_covered_spanners(split)
...         score[x].extend( copied )
...     for x in silenceStaffs:
...         duration = sum( [y.written_duration for y in split] ) 
...         duration = Duration( duration )
...         leaves = leaftools.make_tied_leaf(Rest, duration)
...         score[x].extend( leaves )
... \end{lstlisting}

\caption{From single staff to rotating staffs. \index{From single staff to rotating staffs.}} 
\end{figure}

Sometimes the pitch material varies within the rotation; for example, each of the three playing cellos might play a different pitch, rather than all three playing the same pitch. It becomes necessary to select different pitches for the different cellos performing at any given moment. Because the section of the score has already been modeled as a distributed single staff, another stage is added to the process of rotation, in which the single staff may be dynamically repitched when it is copied to multiple staves:

\begin{figure}[H]
\begin{lstlisting}[basicstyle=\scriptsize\ttfamily, breaklines=True, tabsize=4, showtabs=false, showspaces=false]
>>> def add_repitched_split_to_score_by_tuple(split, score, staffIndexTuple, pitchGroup):       
...     shuffle(pitchGroup)
...     cyclicPitchTuple = sequencetools.CyclicTuple(pitchGroup)
...         #pair the pitch with each staff.
...     staffPitchPairs = zip(staffIndexTuple,cyclicPitchTuple)
...         #(3, 1), (4,-3), (5, 1)
...     #use set difference to get two lists: which staffs have music, and which silence.
...     allStaffs = set(range( len(score) ) )
...     soundStaffs = set(staffIndexTuple)
...     silenceStaffs = allStaffs - soundStaffs
...     for pair in staffPitchPairs:
...         staffIndex = pair[0]
...         pitch = pair[1]
...         copied = componenttools.copy_components_and_covered_spanners(split)
...         copied = repitch_copy(copied, pitch)
...         score[ staffIndex ].extend( copied )
...     for x in silenceStaffs:
...         duration = sum( [y.written_duration for y in split] ) 
...         duration = Duration( duration )
...         leaves = leaftools.make_tied_leaf(Rest, duration)
...         score[x].extend( leaves )
... 
>>> def add_splits_to_score_by_tuples(score, splitTuplePairs, pitchGroup):
...     for pair in splitTuplePairs:
...         staffTuple = pair[0]
...         split = pair[1]
...         if pitchGroup == [ ]:
...             add_split_to_score_by_tuple(split, score, staffTuple)
...         else:
...             add_repitched_split_to_score_by_tuple(split, score, staffTuple, pitchGroup)
... \end{lstlisting}

\caption{Repitching the staff as it rotates. \index{Repitching the staff as it rotates.}} 
\end{figure}

Because the rotation occurred at a resolution of a sixteenth note, rests must be fused in order to create durationally appropriate rests (rather than successions of many sixteenth-note rests.) Beam spanners are applied to beats containing notes:

\begin{figure}[H]
\begin{lstlisting}[basicstyle=\scriptsize\ttfamily, breaklines=True, tabsize=4, showtabs=false, showspaces=false]
>>> def fuse_rests_in_beat(beat):
...     for group in componenttools.yield_topmost_components_grouped_by_type(beat):
...         if isinstance(group[0], Rest):
...             leaftools.fuse_leaves( group[:] )
... 
>>> def fuse_rests_in_staff_by_beats(beats):
...     for beat in beats:
...         fuse_rests_in_beat(beat)
... 
>>> def apply_beam_spanner_to_non_rest_beat(beat, brokenBeam = False):
...     if not all( [isinstance(x,Rest) for x in beat] ):
...         beamtools.BeamSpanner(beat[:],Up)
...         if len(beat) == 4 and brokenBeam == True:
...             right = marktools.LilyPondCommandMark("set stemRightBeamCount = #1")(beat[1])
...             left = marktools.LilyPondCommandMark("set stemLeftBeamCount = #1")(beat[2])
... 
>>> def apply_beam_spanner_to_non_rest_beats(beats, brokenBeam = False):
...     for beat in beats:
...         apply_beam_spanner_to_non_rest_beat(beat, brokenBeam = brokenBeam)
... \end{lstlisting}

\caption{Modeling the low-level typographical habits in the rotation section. \index{Modeling the low-level typographical habits in the rotation section.}} 
\end{figure}

All of this code so far has modeled high-level behaviors of material without considering the selection of pitch. The ability to choose pitch from a reservoir of possible pitches, with and without repetition, is a central element of the music; here a function makes a random choice from a set and the iterative use of this function in a second function creates a staff of randomly chosen pitches:

\begin{figure}[H]
\begin{lstlisting}[basicstyle=\scriptsize\ttfamily, breaklines=True, tabsize=4, showtabs=false, showspaces=false]
>>> def choose_pitch_without_repetition(pitch, choices):
...     chosen = pitch
...     while chosen == pitch:
...         candidate = choice(choices)
...         if candidate != pitch:
...             chosen = candidate
...     return chosen
... 
>>> def make_staff_with_random_pitches(choices, numPitches, brokenBeam = False):
...     notes = [ ]
...     pitchList = choices
...     chosen = choices[-1]
...     for x in range(numPitches):
...         pitch = choose_pitch_without_repetition(chosen, choices)
...         chosen = pitch
...         note = Note(pitch, Duration(1,16) )
...         notes.append( note )
...     staff = Staff(notes)
...     #if brokenBeam == True:
...     #   beats = componenttools.split_components_at_offsets(staff.leaves, [Duration(1,4)], cyclic=True, tie_split_notes=False)
...     #   apply_beam_spanner_to_non_rest_beats(beats, brokenBeam = False)
...     return staff
... \end{lstlisting}

\caption{Function for creating a staff of random pitches. \index{Function for creating a staff of random pitches.}} 
\end{figure}

Finally, another set of typographical functions fuses rests and makes uniform the position of short rests on each staff:

\begin{figure}[H]
\begin{lstlisting}[basicstyle=\scriptsize\ttfamily, breaklines=True, tabsize=4, showtabs=false, showspaces=false]
>>> def beam_and_fuse_beats_in_score_by_durations(score, durations, cyclic=False, brokenBeam = False):
...     for staff in score:
...         beats = componenttools.split_components_at_offsets(staff.leaves, durations, cyclic=cyclic)
...         fuse_rests_in_staff_by_beats(beats)
...         beats = componenttools.split_components_at_offsets(staff.leaves, durations, cyclic=cyclic)
...         apply_beam_spanner_to_non_rest_beats(beats, brokenBeam = brokenBeam)
... 
>>> def fuse_consecutive_rests_of_duration_by_duration_threshold(run, duration, durationThreshold):
...     toFuse = [x for x in run if x.written_duration == duration]
...     runDuration = componenttools.sum_duration_of_components(run[:])
...     if durationThreshold <= runDuration:
...         leaftools.fuse_leaves(toFuse)
... 
>>> def fuse_rests_of_duration_in_bar_by_duration_threshold(bar, duration, durationThreshold):
...     for run in componenttools.yield_topmost_components_grouped_by_type(bar):
...         fuse_consecutive_rests_of_duration_by_duration_threshold(run, duration, durationThreshold )
... 
>>> def fuse_rests_of_duration_in_bars_by_duration_threshold(bars, duration, durationThreshold):
...     for bar in bars:
...         fuse_rests_of_duration_in_bar_by_duration_threshold(bar, duration, durationThreshold )
... 
>>> def fuse_rests_of_duration_in_score_by_duration_threshold(score, duration, durationThreshold):
...     for staff in score:
...         bars = componenttools.partition_components_by_durations_exactly(staff.leaves, [Duration(4,4)], cyclic=True)
...         fuse_rests_of_duration_in_bars_by_duration_threshold(bars, duration, durationThreshold)
... 
>>> def set_vertical_positioning_pitch_on_rests_in_staff(staff, pitch):
...     for rest in iterationtools.iterate_rests_in_expr(staff):
...         resttools.set_vertical_positioning_pitch_on_rest(rest, pitch)
... \end{lstlisting}

\caption{The final stage of typographical adjustment for the rotation section. \index{The final stage of typographical adjustment for the rotation section.}} 
\end{figure}
All of the previously defined functions work together in a function that rotates an arbitrary staff of music around a score of x staffs, placing the music on y staffs at any given moment, with keyword arguments that allow bookending, repitching according to a given pitch set, and an arbitrary phase offset to start midway through rotations:
\begin{figure}[H]
\begin{lstlisting}[basicstyle=\scriptsize\ttfamily, breaklines=True, tabsize=4, showtabs=false, showspaces=false]
>>> def rotate_expression_through_adjacent_staffs_at_bandwidth_by_durations(expression, score, staffIndexBoundsTuple, rotationBandwidth, durations, compressedReflections=True, cyclic=False, phaseOffset= 0, soloBookends = False, brokenBeam = False, pitchGroup = [ ]):
...     matrix = make_cyclic_matrix_for_rotation_by_bandwidth( staffIndexBoundsTuple, rotationBandwidth, compressedReflections )
...     splits = componenttools.split_components_at_offsets(expression.leaves, durations, cyclic=cyclic, tie_split_notes=False)
...     splitTuplePairs = pair_pitches_with_splits(matrix, splits, phaseOffset, soloBookends)
...     add_splits_to_score_by_tuples(score, splitTuplePairs, pitchGroup)
...     beam_and_fuse_beats_in_score_by_durations(score, [Duration(1,4)],cyclic=True, brokenBeam = brokenBeam)
... \end{lstlisting}

\caption{The final rotation function. \index{The final rotation function.}} 
\end{figure}

Utility functions enable this rotation function to put material onto the score:

\begin{figure}[H]
\begin{lstlisting}[basicstyle=\scriptsize\ttfamily, breaklines=True, tabsize=4, showtabs=false, showspaces=false]
>>> def make_empty_cello_score(numStaffs):
...     score = Score([])
...     for x in range(numStaffs):
...         score.append( Staff([]) )
...         contexttools.ClefMark('bass')(score[x])
...         score[x].override.beam.damping = "+inf.0"
...     return score
... 
>>> def add_expression_to_staffs_in_score_by_index_tuple(expr, score, indexTuple, brokenBeam = False):
...     for x in range(indexTuple[0], indexTuple[1]):
...         copies = componenttools.copy_components_and_covered_spanners(expr[:])
...         beats = componenttools.split_components_at_offsets(copies, [Duration(1,4)], cyclic = True)
...         apply_beam_spanner_to_non_rest_beats(beats, brokenBeam = brokenBeam)
...         score[x].extend( componenttools.copy_components_and_covered_spanners(expr[:]) )
... \end{lstlisting}

\caption{Utility functions enable rotation. \index{Utility functions enable rotation.}} 
\end{figure}

At this point, it becomes straightforward to model the succession of events in the score, using the previous functions. First, the program models the first section of music, in which a unison figure rotates throughout the ensemble:
\begin{figure}[H]
\begin{lstlisting}[basicstyle=\scriptsize\ttfamily, breaklines=True, tabsize=4, showtabs=false, showspaces=false]
>>> def make_rotating_unison_section(score):
...     unison = make_staff_with_random_pitches([-3, -1, 1], 24)    
...     add_expression_to_staffs_in_score_by_index_tuple(unison[:-8], score, (0,12) )   
...     add_expression_to_staffs_in_score_by_index_tuple(unison[-8:], score, (0,1) )
...     for staff in score[1:]:
...         staff.append( Rest("r2") )
...     firstRotationPitches = make_staff_with_random_pitches([-3, -1, 1], 64)
...     rotate_expression_through_adjacent_staffs_at_bandwidth_by_durations(firstRotationPitches, score, (0,6), 3, [Duration(1,16)], compressedReflections=True, cyclic=True, brokenBeam = True)
...     for staff in score:
...         staff.extend("b2 ~ b1 ~ b ~ b2.")
...     #m9b4
...     secondRotationLowerPitches = make_staff_with_random_pitches([-3, -1, 1], (13*4) + 6  )  
...     rotate_expression_through_adjacent_staffs_at_bandwidth_by_durations(secondRotationLowerPitches, score, (6,12), 3, [Duration(1,16)], compressedReflections=True, cyclic=True, phaseOffset = 5, soloBookends = True, brokenBeam = True)
...     #m13b3
...     for x in range(6):
...         del(score[x][-3:])
...         score[x].append( "r4" )
...     score[5].pop(-1)
...     score[5].append("r8. b16")
...     secondRotationHigherPitches = make_staff_with_random_pitches([-3, -1, 1], 56  ) 
...     rotate_expression_through_adjacent_staffs_at_bandwidth_by_durations(secondRotationHigherPitches, score, (0,6), 3, [Duration(1,16)], compressedReflections=True, cyclic=True, phaseOffset = 5, brokenBeam = True)
...     for x in range(6,12):
...         del( score[x][-2:] )
...         score[x].append( "r2" )
...         score[x].append( "r16" * 14 )
...     #m16b3
...     add_expression_to_staffs_in_score_by_index_tuple(unison, score, (0,6), brokenBeam = True )
... \end{lstlisting}

\caption{Modeling the score's first rotation with the rotation function. \index{Modeling the score's first rotation with the rotation function.}} 
\end{figure}

In the next section, the rotation continues with each moment sounding all three pitches; the rotation function still applies, this time with an optional pitch set argument:

\begin{figure}[H]
\begin{lstlisting}[basicstyle=\scriptsize\ttfamily, breaklines=True, tabsize=4, showtabs=false, showspaces=false]
>>> def make_rotating_pitch_group_section(score):
...     thirdRotationLowerPitches = make_staff_with_random_pitches([-3, -1, 1], 38  )   
...     rotate_expression_through_adjacent_staffs_at_bandwidth_by_durations(thirdRotationLowerPitches, score, (6,12), 3, [Duration(1,16)], compressedReflections=True, cyclic=True, phaseOffset = 0, soloBookends = True, brokenBeam = False, pitchGroup = [ -3, -1, 1 ] )
...     for x in range(0,6):
...         del(score[x][-2:])
...         score[x].append( "r16" )
...     thirdRotationUpperPitches = make_staff_with_random_pitches([-3, -1, 1], 55  )   
...     rotate_expression_through_adjacent_staffs_at_bandwidth_by_durations(thirdRotationUpperPitches, score, (0,6), 3, [Duration(1,16)], compressedReflections=True, cyclic=True, phaseOffset = 5, soloBookends = True, brokenBeam = False, pitchGroup = [ -3, -1, 1 ] )
...     for x in range(6,12):
...         del(score[x][-1])
... \end{lstlisting}

\caption{Modeling a rotation with multiple simultaneous pitches. \index{Modeling a rotation with multiple simultaneous pitches.}} 
\end{figure}

Next, the entire ensemble plays a tutti unison; because no rotation occurs, the random selection functions alone model this section of the score:

\begin{figure}[H]
\begin{lstlisting}[basicstyle=\scriptsize\ttfamily, breaklines=True, tabsize=4, showtabs=false, showspaces=false]
>>> def generate_random_sixteenth_note():
...     pitches = [ -3, -1, 1]
...     chosen = choice(pitches)
...     return Note( chosen, (1,16) )
... 
>>> def generate_n_random_sixteenth_notes(n):
...     notes = [ ]
...     for x in range(n):
...         note = generate_random_sixteenth_note()
...         notes.append( note )
...     return notes
... 
>>> def add_n_random_sixteenth_notes_to_staff(n, staff):
...     notes = generate_n_random_sixteenth_notes(n)
...     staff.extend( notes )
... 
>>> def add_n_sixteenth_rests_to_staff(n, staff):
...     for x in range(n):
...         staff.extend( "r16" )
... 
>>> def add_n_random_sixteenth_notes_to_staffs_in_score_by_index_tuple(n, score, staffTuple):
...     staffIndexes = range(staffTuple[0], staffTuple[1])
...     staffSet = set(staffIndexes)
...     allStaffs = set( range(len(score)) )
...     silentSet = allStaffs - staffSet
...     for staffIndex in staffIndexes:
...         add_n_random_sixteenth_notes_to_staff(n, score[ staffIndex ] )
...     for staffIndex in silentSet:
...         add_n_sixteenth_rests_to_staff(n, score[ staffIndex ])
... \end{lstlisting}

\caption{Modeling the tutti section with randomly selected sixteenth notes. \index{Modeling the tutti section with randomly selected sixteenth notes.}} 
\end{figure}

Using these functions, the tutti section can be encapsulated into a single function:

\begin{figure}
\begin{lstlisting}[basicstyle=\scriptsize\ttfamily, breaklines=True, tabsize=4, showtabs=false, showspaces=false]
>>> def make_tutti_pitch_group_section(score):
...     rightSide = (0,6)
...     leftSide = (6,12)
...     tutti = (0, 12)
...     add_n_random_sixteenth_notes_to_staffs_in_score_by_index_tuple(40, score, tutti )
...     add_n_random_sixteenth_notes_to_staffs_in_score_by_index_tuple(6, score, rightSide )
...     add_n_random_sixteenth_notes_to_staffs_in_score_by_index_tuple(5, score, leftSide )
...     add_n_random_sixteenth_notes_to_staffs_in_score_by_index_tuple(4, score, rightSide )
...     add_n_random_sixteenth_notes_to_staffs_in_score_by_index_tuple(6, score, leftSide )
...     add_n_random_sixteenth_notes_to_staffs_in_score_by_index_tuple(31, score, tutti )
...     beam_and_fuse_beats_in_score_by_durations(score, [Duration(1,4)],cyclic=True)
... \end{lstlisting}

\caption{The tutti section as single function. \index{Modeling the tutti section with randomly selected sixteenth notes.}} 
\end{figure}

The first ``diatonic'' section of the score, comprised of the first twenty-nine measures of music, may now be encapsulated in a single function:

\begin{figure}[H]
\begin{lstlisting}[basicstyle=\scriptsize\ttfamily, breaklines=True, tabsize=4, showtabs=false, showspaces=false]
>>> def make_diatonic_section(score):
...     make_rotating_unison_section(score)
...     make_rotating_pitch_group_section(score)
...     make_tutti_pitch_group_section(score)
... \end{lstlisting}

\caption{The first section of the score as a single encapsulation. \index{The first section of the score as a single encapsulation.}} 
\end{figure}
The next section of the music appears to be based on a random walk. It has been modeled using a weighted choice from a list of possible intervals; the probability weights were adjusted in a process of trial and error to approximate the published result:

\begin{figure}[H]
\begin{lstlisting}[basicstyle=\scriptsize\ttfamily, breaklines=True, tabsize=4, showtabs=false, showspaces=false]
>>> def make_weighted_choice_from_list( theList ): #makes a weighted choice (by Kevin Parks at snippets.dzone.com)
...     n = uniform(0, 1)
...     for item, weight in theList:
...         if n < weight:
...             break
...         n = n - weight
...     return item
... 
>>> def choose_interval_change_from_weights(pitchWeights):
...     interval = make_weighted_choice_from_list(pitchWeights)
...     octave = make_weighted_choice_from_list( [(0, .95), (1, .05)] )
...     if octave:
...         interval += 8
...     down = make_weighted_choice_from_list( [(0, .5), (1, .5)] )
...     if down:
...         interval = interval * -1
...     return interval
... 
>>> def choose_pitch_based_on_previous_pitch(previous, changeWeights, pitchWeights):
...     change = make_weighted_choice_from_list( changeWeights )
...     if change:
...         chosen = choose_interval_change_from_weights(pitchWeights)
...         return chosen
...     else:
...         return previous
... \end{lstlisting}

\caption{Weighted probability choice functions for the first random walk section. \index{Weighted probability choice functions for the first random walk section.}} 
\end{figure}

Because the random pitch walk might cause a given part to wander lower or higher than the perceptual boundaries of its initial clef's ledger lines, the program reviews the pitches resulting from the selected intervals to automate clef switching:

\begin{figure}[H]
\begin{lstlisting}[basicstyle=\scriptsize\ttfamily, breaklines=True, tabsize=4, showtabs=false, showspaces=false]
>>> def make_from_to_interval_tuples_from_expr(expr):
...     tuples = [ ]
...     for x in range( len(expr) -1):
...         fromPitch = expr[x].written_pitch
...         toPitch = expr[x+1].written_pitch
...         interval = pitchtools.calculate_melodic_chromatic_interval(fromPitch, toPitch)
...         tuple = (fromPitch.chromatic_pitch_number, toPitch.chromatic_pitch_number, interval)
...         tuples.append( tuple )
...     return tuples
... 
>>> def check_for_change_to_treble(tuple):
...     fromPitch = tuple[0]
...     toPitch = tuple[1]
...     interval = tuple[2]
...     if interval.number >= 5 and toPitch >= 5:
...         return 'treble'
...     else:
...         return False
... 
>>> def check_for_change_to_bass(tuple):
...     fromPitch = tuple[0]
...     toPitch = tuple[1]
...     interval = tuple[2]
...     if interval.number <= -5 and -5 >= toPitch:
...         return 'bass'
...     else:
...         return False
... 
>>> def check_to_from_interval_tuple_for_clef_add(x, tuple, staff):
...     effectiveClef = contexttools.get_effective_clef(staff[x])
...     if effectiveClef.clef_name == 'bass':
...         clef = check_for_change_to_treble(tuple)
...     else:
...         clef = check_for_change_to_bass(tuple)
...     return clef
... \end{lstlisting}

\caption{Conditional checks to determine a clef change. \index{Conditional checks to determine a clef change.}} 
\end{figure}

The above functions can then be applied to an expression to add clef changes:

\begin{figure}[H]
\begin{lstlisting}[basicstyle=\scriptsize\ttfamily, breaklines=True, tabsize=4, showtabs=false, showspaces=false]
>>> def add_clef_if_needed(x, tuple, staff):
...     clef = check_to_from_interval_tuple_for_clef_add(x, tuple, staff) 
...     if clef:
...         contexttools.ClefMark(clef)( staff[x+1] )
... 
>>> def add_clefs_to_expr(expr):
...     differenceTuples = make_from_to_interval_tuples_from_expr(expr) 
...     for x,tuple in enumerate(differenceTuples):
...         add_clef_if_needed(x, tuple, expr)
... 
>>> def add_clefs_to_exprs(staffs):
...     for staff in staffs:
...         add_clefs_to_expr(staff)
... \end{lstlisting}

\caption{Applying automatic clef changes to an expression. \index{Applying automatic clef changes to an expression.}} 
\end{figure}

Then the program adds random walk notes to the staffs in the score:

\begin{figure}[H]
\begin{lstlisting}[basicstyle=\scriptsize\ttfamily, breaklines=True, tabsize=4, showtabs=false, showspaces=false]
>>> def choose_pitch_values_from_weights(seedPitch, numNotes, changeWeights, pitchWeights):
...     pitches = [ ]
...     previous = seedPitch
...     for x in range(numNotes):
...         chosen = choose_pitch_based_on_previous_pitch(previous, changeWeights, pitchWeights)
...         pitches.append( chosen )
...         previous = chosen
...     return pitches
... 
>>> def make_n_notes_from_random_pitch_walk(seedPitch, numNotes, changeWeights, pitchWeights):
...     pitches = choose_pitch_values_from_weights(seedPitch, numNotes, changeWeights, pitchWeights)
...     notes = [ Note( x,(1,16) ) for x in pitches]
...     return notes
... 
>>> def add_components_from_staffs_to_score(staffs, score):
...     for x,staff in enumerate(staffs):
...         copied = componenttools.copy_components_and_covered_spanners(staff.leaves)
...         score[x].extend(copied)
...     staff = Staff([])
...     contexttools.ClefMark('bass')(staff[0])
...     add_clefs_to_exprs( [staff] )
...     return staff
... \end{lstlisting}

\caption{Adding random walk notes to the score. \index{Adding random walk notes to the score.}} 
\end{figure}

The random walk sections are entered and exited via unison drones which may or may not be inflected with trills. The following functions query whether or not a trill spanner has been attached to a given score component and fuse trill spanners attached to adjacent components:

\begin{figure}[H]
\begin{lstlisting}[basicstyle=\scriptsize\ttfamily, breaklines=True, tabsize=4, showtabs=false, showspaces=false]
>>> def trill_spanner_attached_to_component(component):
...     spanners = spannertools.get_spanners_attached_to_component( component )
...     for x in spanners:
...         if isinstance(x, spannertools.TrillSpanner):
...             return True
...     return False
... 
>>> def fuse_trill_spanners_attached_to_components(left, right):
...     leftTrill = spannertools.get_the_only_spanner_attached_to_component(left, klass=spannertools.TrillSpanner)
...     rightTrill = spannertools.get_the_only_spanner_attached_to_component(right, klass=spannertools.TrillSpanner)
...     leftTrill.fuse(rightTrill)
... \end{lstlisting}

\caption{Querying and fusing trill spanners. \index{Querying and fusing trill spanners.}} 
\end{figure}

Next, three functions add three respective materials to a specified staff --- an initial drone (preceding the random walk), a random walk, or a terminal drone (following the random walk):

\begin{figure}[H]
\begin{lstlisting}[basicstyle=\scriptsize\ttfamily, breaklines=True, tabsize=4, showtabs=false, showspaces=false]
>>> def add_in_drone_to_staff(staff, inDroneDuration, trill_in, tie_to_previous):
...     startLength = len(staff)
...     endOfBeginningIndex = startLength - 1
...     beginningOfEndIndex = startLength
...     inDrone = leaftools.make_tied_leaf(Note, inDroneDuration, pitches = pitchtools.NamedChromaticPitch("b"))
...     staff.extend( inDrone )
...     if tie_to_previous == True:
...         tietools.apply_tie_spanner_to_leaf_pair( staff[endOfBeginningIndex], staff[beginningOfEndIndex] )
...     if trill_in:
...         inTrillSpanner = spannertools.TrillSpanner(inDrone[:])
...         inTrillSpanner.written_pitch = 2
...     if trill_spanner_attached_to_component( staff[endOfBeginningIndex] ):
...         fuse_trill_spanners_attached_to_components( staff[endOfBeginningIndex], staff[beginningOfEndIndex] )
...     if trill_in and not trill_spanner_attached_to_component( staff[endOfBeginningIndex] ):
...         marktools.LilyPondCommandMark("tieDown")(inDrone[0])
...         marktools.LilyPondCommandMark("tieUp","after")(inDrone[0])
... 
>>> def add_walk_to_staff(staff, walkDuration, changeWeights, pitchWeights):
...     numSixteenths = walkDuration.numerator * (16 / walkDuration.denominator)
...     walkNotes = make_n_notes_from_random_pitch_walk(-1, numSixteenths, changeWeights, pitchWeights)
...     if walkNotes[0].written_pitch < pitchtools.NamedChromaticPitch("c'"):
...         contexttools.ClefMark('bass')(walkNotes[0])
...     else:
...         contexttools.ClefMark('treble')(walkNotes[0])
...     intermediateStaff = Staff(walkNotes)
...     add_clefs_to_expr(intermediateStaff)
...     copied = componenttools.copy_components_and_covered_spanners(intermediateStaff[:])
...     staff.extend(copied)
... 
>>> \end{lstlisting}

\caption{Functions for adding drones and random walks. \index{Functions for adding drones and random walks.}} 
\end{figure}

Just as a single rotation function added a rotation to the score to generate the first section of the score, the previous functions culminate in two functions that add a ``drone and back'' random walk gesture to a staff and score respectively. The ``startEncroachment'' and ``endEncroachment'' arguments specify the number of sixteenth notes of the random walk through which the drone will continue; that is, if a random walk section of three measures has been chosen, the startEncroachment will continue the preceding drone into the first measure by n sixteenth notes and the endEncroachment will begin the subsequent drone n sixteenth notes before the end of the third measure. Other arguments specify the durations of the initial drone, the random walk section, and the following drone:

\begin{figure}[H]
\begin{lstlisting}[basicstyle=\scriptsize\ttfamily, breaklines=True, tabsize=4, showtabs=false, showspaces=false]
>>> def add_drone_and_back_tutti_to_staff(staff, inDroneBaseDuration, startEncroachment, walkBaseDuration, outDroneBaseDuration, endEncroachment, changeWeights, pitchWeights, trill_in=False, trill_out = False, tie_to_previous = False):
...     startLength = len(staff)
...     startEncroachmentAsSixteenths = Duration(startEncroachment, 16)
...     endEncroachmentAsSixteenths = Duration(endEncroachment, 16)
...     inDroneDuration = inDroneBaseDuration - startEncroachmentAsSixteenths
...     if inDroneDuration != 0:
...         add_in_drone_to_staff(staff, inDroneDuration, trill_in, tie_to_previous)
...     walkDuration = walkBaseDuration + startEncroachmentAsSixteenths + endEncroachmentAsSixteenths
...     add_walk_to_staff(staff, walkDuration, changeWeights, pitchWeights)
...     outDroneDuration = outDroneBaseDuration - endEncroachmentAsSixteenths
...     if outDroneBaseDuration != 0:
...         spanner = add_out_drone_to_staff(staff, outDroneDuration, trill_out, tie_to_previous, startLength)
...     if outDroneBaseDuration != 0:
...         return spanner
... 
>>> def add_drone_and_back_tutti_to_score(score, inDroneBaseDuration, startEncroachments, walkBaseDuration, OutDroneBaseDuration, endEncroachments, changeWeights, pitchWeights, trill_in = False, trill_out = False, tie_to_previous = False):
...     startEncroachments = sequencetools.CyclicTuple( startEncroachments )
...     endEncroachments = sequencetools.CyclicTuple(endEncroachments)
...     for x, staff in enumerate(score):
...         spanner = add_drone_and_back_tutti_to_staff(staff, inDroneBaseDuration, startEncroachments[x], walkBaseDuration, OutDroneBaseDuration, endEncroachments[x], changeWeights, pitchWeights, trill_in, trill_out, tie_to_previous)
... \end{lstlisting}

\caption{Functions for adding the random walk gesture to score, framed by drones as specified. \index{Functions for adding the random walk gesture to score, framed by drones as specified.}} 
\end{figure}

The use of this previous function allows a single function to model the random walk section of the score:

\begin{figure}[H]
\begin{lstlisting}[basicstyle=\scriptsize\ttfamily, breaklines=True, tabsize=4, showtabs=false, showspaces=false]
>>> def make_random_walk_section(score):
...     changeWeights = [ (0, .25), (1, .75) ]
...     pitchWeights = [ (.5, .14), (1, .35), (2, .35), (1.5, .14), (3, .01), (4, .01) ]
...     encroachments = range(1,17)
...     endEncroachments = sample(encroachments, 12)
...     add_drone_and_back_tutti_to_score(score, Duration(11,4), [1], Duration(4,1), Duration(2,1), endEncroachments, changeWeights, pitchWeights)
...     encroachments = range(1,33)
...     startEncroachments = sample(encroachments, 12)
...     encroachments = range(1,19)
...     endEncroachments = sample(encroachments, 12)
...     spanner = add_drone_and_back_tutti_to_score(score, Duration(2,1), startEncroachments, Duration(1,1), Duration(2,1), endEncroachments, changeWeights, pitchWeights, trill_out=True, tie_to_previous = True)
...     encroachments = range(0,33)
...     startEncroachments = sample(encroachments, 12)
...     add_drone_and_back_tutti_to_score(score, Duration(2,1), startEncroachments, Duration(1,1), 0, [0], changeWeights, pitchWeights, trill_in=True, tie_to_previous = True)
... \end{lstlisting}

\caption{The random walk section as a single function. \index{The random walk section as a single function.}} 
\end{figure}

Because the previous operations have been executed by adding many small durations to a staff, a metric hierarchy must imposed upon the durations in order for the music to comport with common practice conventions regarding rhythmic division, mainly by fusing chains of many shorter, tied durations into longer durations. (Recent versions of the API have eliminated this step of the process with an object oriented model of metrical hierarchies.) Note that this is the opposite of the previous example, in which longer durations needed to be divided in order to comport with the duration of the piece's meter; the system enables either approach, and the formulation of a procedural strategy must address the relationship between the note as an abstraction and the note as a read symbol that conforms to the conventions of common notation:

\begin{figure}[H]
\begin{lstlisting}[basicstyle=\scriptsize\ttfamily, breaklines=True, tabsize=4, showtabs=false, showspaces=false]
>>> def get_quarter_runs_in_group(group):
...     runs = [ ]
...     run = [ ]
...     for leaf in group:
...         if leaf.written_duration == Duration(1,4):
...             run.append(leaf)
...         else:
...             if run != []:
...                 runs.append(run)
...                 run = [ ]
...     if run != []:
...         runs.append(run)
...     return runs
... 
>>> def get_quarter_runs_in_expr(expr):
...     runs = [ ]
...     for group in componenttools.yield_topmost_components_grouped_by_type(expr):
...         if isinstance(group[0], Note):
...             groupRuns = get_quarter_runs_in_group(group)
...             runs.extend(groupRuns)
...     return runs
... 
>>> def fuse_quarter_runs_in_bar(bar_leaves_in_chain):
...     runs = get_quarter_runs_in_expr(bar_leaves_in_chain)
...     for run in runs:
...         if len(run) > 1:
...             leaftools.fuse_leaves(run[:])
... 
>>> def get_bar_leaves_in_chain(bar, chain):
...     leaves = []
...     for leaf in bar:
...         if bar[0].timespan.start_offset <= leaf.timespan.start_offset and leaf.timespan.stop_offset <= bar[-1].timespan.stop_offset:
...             leaves.append(leaf)
...     return leaves
... 
>>> def fuse_leaves_if_fully_tied(shard):
...     if tietools.are_components_in_same_tie_spanner(shard[:]):
...         leaftools.fuse_leaves(shard)
... 
>>> \end{lstlisting}

\caption{Imposing metric hierarchy by fusing chains of small durations. \index{Imposing metric hierarchy by fusing chains of small durations.}} 
\end{figure}

These functions for fusing culminate in two functions for imposing metric hierarchy on a staff and the entire score, respectively: 

\begin{figure}[H]
\begin{lstlisting}[basicstyle=\scriptsize\ttfamily, breaklines=True, tabsize=4, showtabs=false, showspaces=false]
>>> def fuse_tied_through_quarters_by_bars(chain, bars):
...     for bar in bars:    
...         bar_leaves_in_chain = get_bar_leaves_in_chain(bar, chain)
...         if bar_leaves_in_chain:
...             fuse_tied_through_quarters_in_bar(bar_leaves_in_chain)
... 
>>> def fuse_quarter_runs_by_bars(chain, bars):
...     for bar in bars:    
...         bar_leaves_in_chain = get_bar_leaves_in_chain(bar, chain)
...         fuse_quarter_runs_in_bar(bar_leaves_in_chain)
... 
>>> def clean_up_durations_in_staff(staff):
...     beats = componenttools.split_components_at_offsets(staff.leaves, [Duration(1,4)], cyclic=True)
...     bars = componenttools.split_components_at_offsets(staff.leaves, [Duration(4,4)], cyclic=True)
...     for chain in tietools.iterate_nontrivial_tie_chains_in_expr(staff):
...         fuse_tied_through_quarters_by_bars(chain, bars)
...     bars = componenttools.split_components_at_offsets(staff.leaves, [Duration(4,4)], cyclic=True)
...     for chain in tietools.iterate_nontrivial_tie_chains_in_expr(staff):
...         fuse_quarter_runs_by_bars(chain, bars)
... 
>>> def clean_up_durations_in_score(score):
...     for staff in score:
...         clean_up_durations_in_staff(staff)
... \end{lstlisting}

\caption{Imposing metric hierarchy on the entire score. \index{Imposing metric hierarchy on the entire score.}} 
\end{figure}

The remaining sections of the score could all be modeled similarly: some of the unorthodox notational constructs proved impossible to model using LilyPond without intimate knowledge of the Scheme programming language, which underlies the most basic functions of the typesetting engine. Assuming a valid model of each score section, each section would finally become a constituent line of a function that generates a score object; i. e., with the two previously coded sections:

\begin{figure}[H]
\begin{lstlisting}[basicstyle=\scriptsize\ttfamily, breaklines=True, tabsize=4, showtabs=false, showspaces=false]
>>> def make_windungen_score():
...     score = make_empty_cello_score(12)
...     make_diatonic_section(score)
...     make_random_walk_section(score)
...     clean_up_durations_in_score(score)
...     clean_up_rests_in_score(score)
...     return score
... \end{lstlisting}

\caption{Creating the score object. \index{Creating the score object.}} 
\end{figure}

Finally, the score object can be used to initialize a LilyPondFile object, as seen at the end of the P\"{a}rt example, the attributes of which can be specified to change the layout and formatting properties of the completed document (\ref{sec:windungen}).

\section{Revealed Strengths and Weaknesses of Formalized Score Control}
Although laboriously detailed, this step-by-step description of two modeling tasks makes clear the advantages and disadvantages of this method of notation generation. The P\"{a}rt example shows that the system models low-complexity structures relatively easily, as demonstrated by the small set of generative pitch and rhythm functions that model most of the score; the addition of dynamics and technical markings does not follow a pattern and must be specified in lists, although even these tasks may be automated somewhat with the use of loops, as in the case of rehearsal marks and some technical indications. This suggests that formalized score control most effectively models works with maximum coherence, works that derive their components in an integral way from a minimally small set of generative principles. As the diversity of organizational logics within a work increases, the complexity of a modeling program must scale proportionally. 

Accordingly, the Xenakis example offers a less unanimous view on this method's efficacy. To the modeling process, each shift in texture results in the equivalent of a completely new model of music, and the process of managing the contrasts of one piece becomes the process of modeling many different kinds of musical activity, each of which could result in its own composition. That the system can indeed model each of these varying musical organizations and its accompanying notational comportments is a testament to the flexibility of the system, but one envies the agility with which a more traditional approach to composing facilitates the sudden invention of contrasting material. (It is arguably the case that traditional composition can occur within the system by simply specifying the pitches and durations to be added to score container objects; however, the lack of an underlying model of music/composition guiding the generation of components would locate this in the realm of composition, not analysis/modeling.) In an analysis task, the demand for this kind of rigorous specification leads to a more concrete understanding of which principles of construction might lead to the composition; at the same time, this reveals that traditional musical analysis can be understood as the task of identifying not every single operative constraint, as must be done here, but the most relevant structures and constraints at each moment in a work of music.

It is also the case, from the perspective of musical analysis, that many of the functions here that fuse smaller durations or divide larger durations model the metric conventions of common practice notation rather than symbolic manipulations indigenous to a specific composition. One might argue that these operations remain outside of the realm of analysis, as they ally more closely to a notion of performance practice or notational technology than to the qualities of a work composed within a performance tradition, using a specific notational technology. (They model notation, not music/composition.) One may also argue, in response, that compositional style extends into simple notational choices: the additive rhythms of Olivier Messiaen, for example, extend additive formulations of meter and rhythm into the realm of rhythmic convention by flouting the regular metric division of a bar; likewise, notations that elect to show clearly the way that a beat has been divided communicate through a primarily divisive, rather than additive, view of the relationship between rhythm and meter. Although there might be a disconnect between musical idea and rhythmic convention --- it would be equally possible to convey Messiaen's rhythms with a divisive metric notation --- there is nonetheless a relationship between the two; notational habit countenances compositional thought.

These analytic tasks elide with compositional applications of the system, and the most significant potential of this kind of comprehensive modeling may lie in future compositional applications of musical models derived from analysis: if coding is done with good style, in easily testable and reconfigurable modules, it becomes easy to reuse modeling functions, substituting new values for the function's arguments, in order to create new music, which may or may not resemble the original composition. In this sense, the interplay between composition and analysis, between historical understanding and contemporary creation, has also been formalized, and a unit of code created to understand history can easily be repurposed for the creation of new work.

%
% Of course, if you prefer, you can just start with
%   \chapter{My First Chapter Name}
% and start typing away.  
%\chapter{Just a Test}
%This is only a test.

\include{Chapter1}
\include{Chapter2}
\include{Chapter3}
\include{Chapter4}
%\bibliographystyle{unsrt}  %% This is just my personal favorite style. 
%                              There are many others.
%\bibliography{review}  %% This looks for the bibliography in myrefs.bib 

%\section{A section}
%Lorem ipsum dolor sit amet, consectetuer adipiscing elit. Nulla odio
%sem, bibendum ut, aliquam ac, facilisis id, tellus. Nam posuere pede
%sit amet ipsum. Etiam dolor. In sodales eros quis pede.  Quisque sed
%nulla et ligula vulputate lacinia. In venenatis, ligula id semper
%feugiat, ligula odio adipiscing libero, eget mollis nunc erat id orci.

%
%Nullam ante dolor, rutrum eget, vestibulum euismod, pulvinar at, nibh.
%In sapien. Quisque ut arcu. Suspendisse potenti. Cras consequat cursus
%nulla. "I am going to complain all the time." \cite[pg. 144]{Abdallah2009}
%\subsection{More Stuff}
%Blah
%
%\begin{figure}[h] 
%  \centering
%  *
%  \caption{A figure of Vonnegut.\index{Vonnegut}} 
%\end{figure}



\appendix
\chapter{Code Examples}
\include{CodeAppendix}
\chapter{Score Examples}
\include{ScoreAppendix}
\include{timeline}

\setmonofont{Palatino}
\linespread{1}
\setlength\bibitemsep{2.0\itemsep}
\printbibliography 
%% END MATTER
% \printindex %% Uncomment to display the index
% \nocite{}  %% Put any references that you want to include in the bib 
%               but haven't cited in the braces.
%\bibliographystyle{plain}  %% This is just my personal favorite style. 
%                              There are many others.
%\bibliography{review}  %% This looks for the bibliography in myrefs.bib 
%                          which should be formatted as a bibtex file.
\end{document}

